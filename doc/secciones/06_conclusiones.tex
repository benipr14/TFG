\chapter{Conclusiones y trabajos futuros}

\section{Conclusiones}

En conclusión, este TFG tenía como objetivo la predicción de partidos de fútbol mediante el uso de machine learning, haciendo uso de una base de datos con las estadísticas y mapas de calor de los futbolistas, dándonos una visión espacial del rendimiento de cada futbolista. Se ha conseguido una precisión de hasta un 87.2\%, algo que se puede considerar exitoso ya que el fútbol es un deporte muy impredecible.

Para completar este objetivo general se ha tenido que cumplir con la realización de los objetivos específicos indicados en el capítulo 1:

\begin{itemize}
    \item OE1: Revisión de la literatura. Se ha completado al 100\%, buscando artículos de calidad sobre fútbol y machine learning obteniendo un estado del arte completo y que trata varios ámbitos dentro de este campo. Se puede consultar en el capítulo 2, Estado del arte.

    \item OE2: Diseño de la base de datos. Se ha completado al 100\%, se ha definido una base de datos completa en MongoDB que cubría todas las necesidades del proyecto. Se puede consultar en el capítulo 4, Análisis y diseño.

    \item OE3: Obtención de las estadísticas de los jugadores. Se ha completado al 100\%, se ha implementado un programa para la inserción de los datos y se han sido obtenidos correctamente. Se puede consultar en el capítulo 6, Implentación.

    \item OE4: Diseño del algoritmo. Se ha completado al 100\%, se ha diseñado un algoritmo para determinar el ganador de un partido dividiendo el campo en zonas y utilizando las estadísticas almacenadas de los jugadores, así como ver que equipos ha dominado en cada zona. Se puede consultar en los capítulos 4 y 5, de Análisis e Implementación respectivamente.

    \item OE5: Implementación de machine learning. Se ha completado al 100\%, se han implementado 2 algoritmos genéticos, uno para el vector de pesos y otro para determinar además en que punto cortar para determinar al ganador, obteniendo resultados casi idénticos en ambos. Se puede consultar en el capítulo 6, Implementación

    \item OE6: Obtención del rendimiento individual de cada jugador del partido. Se ha completado al 100\%, se han usado las estadísticas del partido para ver la influencia de cada futbolista de manera individual y obtener así un análisis mucho más profundo para ver que futbolistas son mas diferenciales. Se puede consultar en el capítulo 6, Implementación.
\end{itemize}

En este TFG también se ha apoyado la inclusión del fútbol femenino, formando parte de la base de datos y analizando a algunas jugadoras en partidos concretos como en el capítulo 5, Implementación.

Como valoración personal, estoy satisfecho con este TFG ya que el fútbol es un deporte que me apasiona y que sigo asiduamente. Tampoco me ha costado mucho estar involucrado con el proyecto  por esos motivos, además de que he aumentado mis conocimientos futbolísticos. En cuanto a lo académico, he aprendido sobre los diferentes mecanismos de machine learning, especialmente de los algoritmos genéticos, que nunca los había dado en la carrera. También he aprendido a organizarme mejor en el tiempo y a hacer el código de más calidad y no tan desordenado. En conclusión, estoy satisfecho con el trabajo realizado y con lo aprendido.

\section{Trabajos futuros}
Aunque se hayan cumplido todos los objetivos de este TFG, tanto específicos como el general, todavía hay muchas opciones para continuar desarrollando.

Por ejemplo, en este proyecto se muestra la actuación individual de cada jugador para que el entrenador pueda analizarlo, sin embargo, se podría hacer que una máquina lo hiciera por él, generando la mejor formación y alineación posible según el rival. También se podría hacer más preciso, dividiendo el partido en varias partes de 15 minutos por ejemplo, y analizar el rendimiento del equipo en cada una de ellas. Esto es algo que ocurre a menudo en el fútbol moderno, un entrenador puede ordenar a su centrocampista por la izquierda que haga coberturas durante la primera media hora de partido porque el rival siempre sale presionando por ese lado, y a partir de esa media hora ayudar a construir en el centro del campo. Teniendo en cuenta esto, una máquina podría recomendar la mejor posición de cada jugador en cada período de tiempo para que el entrenador lo ordene en ese partido.

En cuanto a la precisión de la predicción, se podría llegar a simular un partido con agentes, esto es, teniendo en cuenta las estadísticas de los futbolistas como pase, tiro, regate. Simular un partido que refleje unas características parecidas pero siempre con un rango de aleatoriedad. Aunque sí es cierto que esto es algo que ya hacen algunos videojuegos como Top Eleven, \cite{Top_eleven}, que simulan partidos de equipos e incluso se pueden ver a los jugadores moverse durante el partido, pero se podría intentar hacer de una manera mucho más precisa y con fundamento matemático.

Por último, ya que se tienen almacenadas las estadísticas de los jugadores, se podría ampliar la base de datos y hacer un sistema de recomendación de fichajes para los equipos. Por ejemplo, mediante un algoritmo de machine learning se analizan los puntos débiles del equipo, digamos el lateral derecho, y automáticamente busque en la base de datos el mejor lateral derecho que cumpla con las características necesarias. Así, si un equipo que por ejemplo juege por las bandas tiene un lateral que pone malos centros, lo podría detectar y recomendarle uno con una alta tasa de centros. Todo esto lo podría detectar el algoritmo detectando puntos débiles en el conjunto de características del equipo.