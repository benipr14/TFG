\chapter{Estado del arte}

El fútbol es un deporte altamente impredecible, en el que incluso los detalles más pequeños pueden tener un impacto significativo en el resultado de un partido. Dada la complejidad del juego y la cantidad de variables implicadas, resulta esencial identificar qué áreas han sido objeto de estudio mediante técnicas de aprendizaje automático y cuáles presentan aún oportunidades para futuras investigaciones. Al finalizar este trabajo, se contará con una visión general y estructurada sobre los aspectos del fútbol que han sido previamente explorados, así como aquellos que permanecen relativamente inexplorados. Para la recopilación de información, se han utilizado filtros de búsqueda aplicados a términos relacionados con 'football' o 'soccer', combinados con 'machine learning' o 'deep learning'.

Como criterios de inclusión y exclusión se han usado los siguientes:
\begin{itemize}
    \item Individuos: Los participantes analizados deben ser exclusivamente jugadores de fútbol. No obstante, también se aceptan estudios que incluyan múltiples disciplinas deportivas, siempre que se puedan extraer y utilizar únicamente los datos correspondientes al fútbol.
    \item Técnica: Los datos han tenido que ser procesados mediante \textit{machine learning} o \textit{deep learning} y no mediante otras técnicas.
    \item Predicción: La predicción debe centrarse en el rendimiento individual o colectivo de los jugadores, excluyendo cualquier otro ámbito ajeno al mundo del fútbol, como aspectos económicos o políticos.
    \item Idioma: los textos son escritos originalmente en inglés y no en otro idioma, o hechos basados en otros artículos originales.
\end{itemize}

Como puede observarse, los criterios de inclusión y exclusión han sido definidos con el objetivo de centrarse exclusivamente en el ámbito del fútbol y en los datos relativos a los futbolistas, al mismo tiempo que se asegura la selección de estudios de calidad.

\section{Revisión de la literatura}

En el ámbito del fútbol, la investigación basada en datos se ha centrado principalmente en tres grandes áreas: la predicción de lesiones en los futbolistas, la estimación de los resultados de los partidos y el pronóstico del desarrollo de la carrera deportiva de jugadores individuales. Estas categorías agrupan la mayoría de los estudios recientes orientados al análisis y la predicción dentro de este deporte.

Cada una de estas categorías abarca a su vez diferentes enfoques y subdivisiones específicas, las cuales se irán desarrollando a lo largo de esta sección con el objetivo de ofrecer una visión general del estado actual de la investigación en este ámbito.

El primer tema a tratar es el de las lesiones en futbolistas. Diversos estudios han intentado predecir su aparición, identificando los factores más comunes con el objetivo de prevenirlas. Una de las lesiones más frecuentes es la distensión de isquiotibiales. En el estudio realizado por Ayala et al. \cite{first-injury}, se analizaron 96 jugadores profesionales pertenecientes a 18 equipos distintos, teniendo en cuenta variables como la pierna dominante, el estado psicológico del jugador y otros atributos físicos medibles, como la fuerza o la flexibilidad. Utilizando un algoritmo de \textit{machine learning} AD Tree, llegaron a la conclusión de que, con más de un 78\% de probabilidad, sí había factores que eran predominantes en las 18 lesiones que detectaron, entre ellos la calidad del sueño, el historial de lesiones de isquiotibial o el ángulo de su rodilla al golpear el balón.

En este otro estudio de López-Valenciano \textit{et al.} \cite{second-injury}, se analizaron a 98 jugadores de 4 equipos diferentes y se tuvieron en cuenta factores idénticos al anterior. Su objetivo era minimizar una lesión de cualquier músculo del cuerpo, y para ello utilizaron 4 árboles de decisión que fueron C4.5, SimpleCart, ADTree y RandomTree. Los resultados obtenidos indican que, con un 66\% de probabilidad, podrían ayudar a los entrenadores en la reducción de las lesiones de sus jugadores mediante la modificación de los entrenamientos al detectar posibles riesgos en algunos ejercicios determinados. En este estudio de Rommers \textit{et al.} \cite{third-injury}, se analizaron a 734 niños de entre 10 y 15 años durante una temporada entera. Su objetivo era predecir las lesiones por sobrecarga en los músculos, por lo que se tomaron medidas como la altura, peso, velocidad, resistencia, etc. Utilizaron un algoritmo XGBoost y con una precisión entre el 75\% y 85\% pudieron predecir qué jugadores eran los que tenían más riesgo de lesión.

Como se puede apreciar, en el ámbito de las lesiones se han hecho varios estudios, y la mayoría de ellos son exitosos, lo que nos lleva a pensar que es un campo ya explorado, pero a su vez nos indica que hay campos dentro del fútbol que sí se pueden predecir con precisión.

Vamos ahora a analizar aquellos estudios relacionados con la predicción de partidos o de eventos que ocurran dentro de partidos. En este primer estudio de Dick y Brefeld \cite{first-outcome}, se analizaron 5 partidos de ligas de fútbol profesional con gran variedad de ataques, los cuales 380 fueron exitosos y 715 no lo fueron. Su objetivo era predecir qué situaciones del juego son las más favorables para tener un ataque exitoso. Un ataque exitoso se entiende como aquel en el que el balón está a menos de 25 metros de la portería rival. Para ello, utilizan algunos atributos como la posesión del equipo, las posiciones de cada uno de los jugadores con coordenadas X e Y y el tiempo de juego efectivo, utilizando el algoritmo de optimización de Adam (Adaptive Moment Estimation). Finalmente, llegaron a la conclusión de que, aparte de saber qué ataques podrían ser exitosos, también podría utilizarse para evaluar la posición de los jugadores y la velocidad de los ataques. En este otro estudio mucho más grande de Goes \textit{et al.} \cite{second-outcome}, se analizaron a 26 equipos profesionales durante 4 temporadas y contabilizaron 1.237 ataques exitosos y 11.187 que no lo fueron. Mediante un sistema de video semiautomático, analizaron el centro geométrico entre las distintas líneas del equipo, sobre todo del portero con la defensa y la defensa con los centrocampistas. Se utilizó un algoritmo de k-medias, y se llegó a la conclusión de que, aplicando el centro geométrico a las líneas del equipo en vez de al equipo entero en su totalidad, es más preciso a la hora de predecir ataques exitosos, así como que los ataques exitosos dependen de que los defensores creen espacio para los atacantes y de que estén sincronizados con los centrocampistas.

En el estudio presentado por AlMulla \textit{et al.} en \cite{third-outcome}, se analizaron partidos de la liga catarí disputados entre 2011 y 2022, utilizando datos históricos de los jugadores que participaron en dichos encuentros. Mediante el uso de redes neuronales GRU (Gated Recurrent Units), se logró predecir el equipo ganador con una precisión del 80\%. Además, el análisis se extendió al desempeño de los jugadores en distintas fases del partido, concluyéndose que los defensores desempeñan un papel determinante en el resultado final. Asimismo, se identificó que los intervalos temporales comprendidos entre los minutos 15 y 30 de cada mitad son los más significativos en términos de impacto en el rendimiento general del equipo.

Como se puede observar, estos estudios se centran en el análisis de ataques exitosos y en la predicción del equipo ganador. No obstante, se basan únicamente en estadísticas individuales de los jugadores, sin considerar su relación con las zonas del campo en las que han estado presentes.

Ahora vamos a pasar a analizar 2 estudios sobre los pases de los jugadores. En el primero de ellos, de Szczepańsk y McHale \cite{first-passes}, analizan 760 partidos durante 2 temporadas en la primera división inglesa, teniendo en cuenta cuál es la calidad del jugador que da el pase y su situación antes de darlo. Su objetivo es predecir la dificultad del pase y la probabilidad de que este sea exitoso, y para ello utilizan un algoritmo de naive bayes. El estudio es exitoso y consigue detectar quiénes serían los mejores pasadores. En este otro estudio más preciso de Chawla \textit{et al.} \cite{second-passes}, se analizó al equipo de fútbol 'Arsenal' durante 4 partidos de la primera división inglesa para intentar predecir sus pases. Utilizando un sistema de cámaras y de observadores humanos, contabilizaron 2.932 pases durante el análisis. Para la predicción se tuvieron en cuenta factores como la trayectoria que sigue el jugador en el campo, los toques al balón, pases, tiros y mapas donde dominaba más cada jugador. Para el algoritmo utilizaron un RUSBoost classifier y un multinomial logistic regression y con más de un 90\% de precisión consideraron que sí podían predecir cómo de bueno iba a ser un pase que se iba a dar.

En estos estudios se logró predecir con éxito la calidad y efectividad de los pases, considerando atributos del jugador como su posición en el campo. Sin embargo, dicha técnica no se extendió al ámbito de la predicción del resultado de los partidos.

Después se realizaron otros estudios como el de Link y Hoernig \cite{first-other}, donde se analizaron 60 partidos de la liga alemana, con un total de casi 70.000 acciones captadas mediante un sistema de cámaras semiautomático. Tuvieron en cuenta la posición de los jugadores, posesión del equipo, posesión individual y acciones individuales y colectivas con el balón. Querían determinar cuánto tiempo pasa el balón en la esfera de influencia de un jugador, basándose en la distancia entre los jugadores y el balón, junto con su dirección de movimiento, velocidad y aceleración. Se hace mediante la configuración espacio-temporal del jugador que controla el balón utilizando redes bayesianas. El estudio llega a la conclusión de que podemos saber durante cuánto tiempo ha estado el balón en la esfera de influencia de un futbolista y conocer su influencia en el juego.

En este otro estudio de Montoliu \textit{et al.} \cite{second-other}, se analizaron cuatro equipos de la primera división española durante cuatro temporadas y se tuvieron en cuenta durante los ataques del equipo la posesión del balón, los regates, desmarques, así como los libres directos e indirectos. Su objetivo era predecir cuál sería el patrón de ataque del equipo tanto en jugadas de posesión como a balón parado. Utilizaron un algoritmo de random forest y k-vecinos más cercanos y, entre una precisión entre 67\% y 93\%, podían predecir cuál sería el patrón de ataque del equipo. Este otro estudio de Knauf \textit{et al.} \cite{third-other}, también analiza los patrones de un equipo; para ello, analizó 10 partidos de primera y segunda división alemana durante una temporada, y tuvo en cuenta la trayectoria del jugador en el campo y las oportunidades de gol en el último cuarto del campo. Utilizaron un algoritmo k-medoids y finalmente pudieron predecir si los ataques iban a ser lentos o rápidos, así como la cantidad de pases por ataque.

Predecir patrones de ataque parece ser exitoso utilizando \textit{machine learning}, por lo que nos indica que esta es ya otra área explorada del fútbol.

Por otro lado, existen diversos estudios que analizan la posible trayectoria de un futbolista a lo largo de su carrera profesional. Uno de ellos es el trabajo presentado por Barron \textit{et al.} en \cite{first-career}, en el que se estudia una muestra de 966 jugadores: 209 no profesionales, 637 pertenecientes a la segunda división inglesa y 120 a la primera división inglesa. En este estudio se tienen en cuenta variables como el número de pases, su precisión y consistencia, así como entradas, duelos ganados y tiros. El objetivo principal es predecir el nivel futuro del jugador y la división en la que podría competir en la siguiente temporada, utilizando para ello una red neuronal artificial (ANN). Los resultados muestran una precisión de entre el 69\% y el 77\% en la predicción de la división futura y la trayectoria profesional del jugador a medio plazo.

En otro estudio relevante de Ćwiklinski \textit{et al.} \cite{second-career}, se analizaron 4.700 jugadores pertenecientes a 156 equipos de las ocho principales ligas europeas. Este trabajo considera un total de 29 atributos técnicos, variables psicológicas y datos de rendimiento en partidos anteriores. El objetivo es predecir si la transferencia de un jugador entre dos clubes será exitosa, entendiéndose como tal una mejora en el rendimiento del jugador respecto a su temporada anterior. Para ello, se emplean algoritmos como Random Forest, Naive Bayes y AdaBoost. Aunque los resultados permiten estimar el posible éxito de una transferencia, los autores destacan que estos deben interpretarse más como recomendaciones que como predicciones absolutas.

Otra dimensión interesante de predicción en el fútbol es la elaboración de alineaciones. Aunque existen pocos estudios centrados en este aspecto, uno de ellos, hecho por García-Aliaga \textit{et al.} \cite{starting-up}, analiza más de 50.000 partidos y 30.000 jugadores a lo largo de siete temporadas en 18 ligas nacionales distintas. El objetivo del estudio es determinar cuál sería la posición óptima para cada jugador, teniendo en cuenta sus características individuales, con el fin de maximizar su rendimiento. Para ello, se emplea el algoritmo RIPPER (Repeated Incremental Pruning to Produce Error Reduction), logrando una precisión superior al 73\% en la predicción de la posición más adecuada para cada jugador. Estos resultados pueden resultar especialmente útiles a la hora de diseñar alineaciones más eficaces.

Como se ha podido observar, existen numerosos ámbitos en los que el aprendizaje automático se aplica con éxito en el contexto del fútbol. No obstante, aunque algunos estudios mencionan la posibilidad de generar mapas de calor que reflejen la influencia espacial de cada jugador, ninguno de ellos los utiliza directamente como herramienta para la predicción. Del mismo modo, aunque hay investigaciones que intentan predecir el resultado de un partido a partir de ciertos atributos, no se establece ninguna relación con los mapas de calor que muestran las zonas del campo donde ha intervenido cada jugador.

\section{Contribuciones de este trabajo}
El análisis del estado del arte ha permitido identificar múltiples enfoques aplicados al ámbito del fútbol, como el uso de datos estadísticos, modelos de \textit{machine learning} para la predicción de resultados y representaciones visuales como los mapas de calor. No obstante, se ha detectado una carencia significativa de estudios que combinen de forma sistemática las características individuales de los jugadores con su posicionamiento e influencia espacial durante el partido.

En este contexto, las principales contribuciones de este Trabajo de Fin de Grado son las siguientes:

\begin{itemize}
    \item Integración del rendimiento individual y espacial: Se propone un modelo que relaciona los atributos individuales de los jugadores con su distribución en el terreno de juego, utilizando mapas de calor como fuente principal de información contextual.
    \item Aplicación de técnicas de \textit{machine learning} a datos espaciales específicos: Frente a estudios previos centrados en estadísticas generales o resultados de equipo, este trabajo aplica algoritmos de aprendizaje automático sobre datos derivados directamente de la actividad espacial de los futbolistas, con el objetivo de identificar patrones relevantes en relación con el rendimiento y el resultado del partido.
    \item Exploración de un área poco abordada: Al centrarse en el análisis de mapas de calor como herramienta predictiva, este trabajo abre una línea de investigación emergente que ha sido poco explorada en la literatura existente.
    \item Base para desarrollos futuros: La metodología desarrollada, junto con los resultados obtenidos, sienta las bases para futuras ampliaciones, como el aumento del volumen de datos, la incorporación de nuevas variables contextuales o la aplicación a competiciones y niveles distintos.
\end{itemize}

Estas contribuciones permiten situar este trabajo dentro del campo de la analítica deportiva, proporcionando un enfoque novedoso que combina el análisis espacial y estadístico con técnicas avanzadas de predicción.
