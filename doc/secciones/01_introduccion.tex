\chapter{Introducción}

El análisis de datos deportivos ha alcanzado una relevancia creciente en el ámbito del fútbol, donde la complejidad del juego y la gran cantidad de variables involucradas generan un alto grado de incertidumbre en los resultados. Este Trabajo de Fin de Grado (TFG) se centra en abordar dicha incertidumbre mediante el desarrollo de un modelo analítico que permita predecir, de manera post-mortem, el resultado de partidos ya disputados, así como evaluar el rendimiento individual de los jugadores.

A diferencia de las aplicaciones orientadas a predecir resultados futuros, el presente trabajo se basa en el estudio de encuentros ya celebrados con el fin de ofrecer al cuerpo técnico información retrospectiva que facilite el análisis táctico y estratégico. De esta forma, se combinan estadísticas individuales y mapas de calor de cada futbolista para construir una métrica global que refleje la ventaja competitiva de un equipo frente al adversario.

Asimismo, este proyecto también puede resultar de interés para personas ajenas al ámbito profesional del fútbol, como los aficionados que desean anticipar el resultado de un encuentro por mera curiosidad, o incluso aquellos que participan en apuestas deportivas y buscan aumentar sus posibilidades de acierto mediante el análisis estadístico.

\section{Motivación y contexto}

En la actualidad, el fútbol posee una notable relevancia social, con capacidad incluso de influir en el estado emocional de las personas. Según Janhub \cite{impact-football-mood}, las victorias o derrotas de un equipo inciden significativamente en el ánimo de sus seguidores, especialmente cuando asisten al estadio, generando una mayor felicidad en caso de triunfos inesperados. Esta carga emocional incrementa la presión sobre jugadores y entrenadores, dado que las expectativas son elevadas, independientemente del nivel competitivo. En esta línea, Özsari \textit{et al.} \cite{act-footballist} señalan que el rendimiento individual de un futbolista repercute directamente en sus relaciones interpersonales y en su bienestar cotidiano: cuanto mejor actúe en el terreno de juego, mayor será su satisfacción personal.

Otro aspecto a tener en cuenta en el contexto del fútbol es su impacto económico. Tal como expone Aygün \textit{et al.} en el estudio \cite{economy-football}, existe una relación directa entre la evolución del fútbol y el desarrollo económico, especialmente a nivel local. Ambos factores presentan un crecimiento correlativo y se influyen mutuamente, lo que pone de manifiesto que el fútbol no solo tiene una función social destacada, sino que también actúa como motor económico en muchas comunidades.

En este sentido, el fútbol puede ejercer una influencia considerable en la vida de muchas personas. Como expresó el exfutbolista Jorge Valdano, 'el fútbol es la cosa más importante entre las menos importantes', reflejando la paradoja entre su aparente trivialidad y su profundo impacto social. En un contexto donde este deporte adquiere tal relevancia, los equipos buscan optimizar al máximo sus resultados, ya sea a través de estrategias dentro del terreno de juego, como la gestión del tiempo o las protestas al árbitro, o mediante el análisis detallado de datos propios y del rival. En un deporte tan competitivo, incluso el más mínimo detalle puede marcar la diferencia entre la victoria y la derrota.

Si bien el fútbol comprende diversas modalidades, como partidos entre selecciones o clubes, así como encuentros amistosos u oficiales, este estudio se centrará exclusivamente en partidos oficiales disputados por clubes. Por tanto, toda referencia a un 'partido' se referirá a encuentros correspondientes a ligas domésticas, pudiendo ser tanto en categoría masculina como femenina. En lo que respecta a los mapas de calor, se entenderán como representaciones gráficas del terreno de juego que muestran las zonas en las que un jugador ha intervenido con mayor frecuencia. Las áreas de mayor intensidad indican una concentración más alta de acciones, lo que permite identificar las regiones del campo donde cada futbolista tiene una presencia más destacada.

En el contexto de la creciente demanda por obtener resultados deportivos positivos, ha emergido una línea de investigación prometedora: la predicción de partidos mediante técnicas de \textit{machine learning}. Esta disciplina permite analizar grandes volúmenes de datos, como los mapas de calor de los jugadores, para identificar patrones de comportamiento sobre el terreno de juego. Esta información puede ser de gran utilidad para los entrenadores a la hora de tomar decisiones estratégicas, como la elección de la alineación más adecuada en función de las zonas del campo donde cada jugador tiene mayor impacto.

A pesar del potencial de esta metodología, los estudios que combinan mapas de calor y \textit{machine learning} en el ámbito del fútbol son todavía escasos, lo que convierte a esta área en un campo de investigación emergente. Cuanto mayor sea el volumen y la calidad de los datos disponibles, más precisas podrán ser las predicciones generadas por los modelos. Por ello, una mejora futura relevante sería la ampliación de la base de datos utilizada, incorporando más encuentros y variables que permitan afinar los resultados.

No obstante, es importante considerar la naturaleza impredecible del fútbol. Factores aleatorios o situaciones puntuales pueden alterar el desarrollo esperado de un partido, limitando así la capacidad de predicción. Aunque los modelos estadísticos y de \textit{machine learning} pueden estimar probabilidades con cierto grado de precisión, siempre existirá un margen de incertidumbre. Aun así, optimizando estos algoritmos, es posible extraer información valiosa tanto sobre el rendimiento individual como colectivo, lo que podría facilitar la labor del cuerpo técnico en la toma de decisiones orientadas a maximizar las posibilidades de victoria.

\section{Objetivos}
El objetivo principal de este TFG es la predicción de partidos de fútbol a través de las estadísticas de los futbolistas y de las zonas del campo en las que han estado mediante \textit{machine learning}.

Para conseguir este objetivo principal se ha realizado el cumplimiento de los siguientes objetivos específicos:
\begin{itemize}

  \item OE1: Diseñar una métrica global que combine estadísticas individuales y distribución espacial de la actividad de los jugadores para cuantificar la ventaja competitiva de un equipo.
  \item OE2: Optimizar los parámetros de dicha métrica mediante técnicas de \textit{machine learning}.
  \item OE3: Validar la fiabilidad y robustez del modelo con una base de datos independiente de más de 200 partidos ya disputados.
  \item OE4: Generar mapas de calor comparativos que permitan evaluar la influencia individual de los futbolistas y facilitar el análisis táctico.
  \item OE5: Proponer directrices prácticas para el cuerpo técnico basadas en los resultados del modelo, orientadas a mejorar la estrategia de juego y la selección de alineaciones.
    
\end{itemize}

\section{Estructura de la memoria}
Este Trabajo de Fin de Grado (TFG) se organiza en 6 capítulos que parten desde la relevancia social del fútbol hasta la implementación de modelos de \textit{machine learning} basados en mapas de calor. La estructura es la siguiente:

\begin{itemize}
    \item Introducción: En este apartado se contextualiza el impacto del fútbol en la sociedad actual, destacando la creciente importancia de su análisis y predicción. Asimismo, se exponen los objetivos principales que se pretenden alcanzar a lo largo del proyecto.
    \item Estado del arte: Se presenta una revisión bibliográfica sobre los estudios previos relacionados con el fútbol en distintas áreas, tanto desde el punto de vista técnico como analítico. A partir de este análisis, se justifica la línea de desarrollo adoptada en el proyecto, identificando oportunidades de mejora y enfoques poco explorados.
    \item Planificación, metodología y presupuesto del proyecto: Se expone la metodología ágil SCRUM empleada para gestionar el desarrollo del proyecto, detallando su división en iteraciones. Además, se presenta la planificación temporal y se incluye una estimación del presupuesto necesario, considerando los recursos utilizados y el coste asociado al desarrollo.
    \item Análisis del problema y diseño de la solución: En este capítulo se define el problema a abordar y se describe el enfoque adoptado para resolverlo. Se incluyen las formulaciones matemáticas necesarias, así como el diseño general del sistema propuesto.
    \item Implementación: Se describe el proceso de desarrollo de la solución propuesta, detallando las decisiones técnicas, las herramientas utilizadas y la forma en que se integraron los distintos componentes del sistema.
    \item Análisis de resultados: Se presentan y analizan los resultados obtenidos tras aplicar la solución desarrollada, evaluando su rendimiento mediante métricas específicas y discutiendo su validez en el contexto del problema abordado.
    \item Conclusiones y trabajos futuros: Finalmente, se presentan las principales conclusiones del trabajo y se proponen posibles líneas de investigación futura que podrían complementar o ampliar los resultados obtenidos.
\end{itemize}

\section{Licencia de código abierto}
El código desarrollado en el marco de este trabajo se encuentra disponible en el siguiente repositorio público de GitHub: \url{https://github.com/benipr14/TFG/tree/master} Este código ha sido liberado bajo los términos de la licencia MIT, permitiendo su uso, modificación y distribución, siempre que se incluya el aviso de copyright correspondiente y una copia de la licencia.

No obstante, todos los datos empleados en este trabajo han sido obtenidos de la página web de Sofascore \cite{Sofascore}. Cabe señalar que dichos datos están protegidos por derechos de autor y no han sido liberados para su uso público ni académico.
