%%%%%%%%%%%%%%%%%%%%%%%%%%%%%%%%%%%%%%%%%
% Short Sectioned Assignment LaTeX Template Version 1.0 (5/5/12)
% This template has been downloaded from: http://www.LaTeXTemplates.com
% Original author:  Frits Wenneker (http://www.howtotex.com)
% License: CC BY-NC-SA 3.0 (http://creativecommons.org/licenses/by-nc-sa/3.0/)
%%%%%%%%%%%%%%%%%%%%%%%%%%%%%%%%%%%%%%%%%

% \documentclass[paper=a4, fontsize=11pt]{scrartcl} % A4 paper and 11pt font size
\documentclass[11pt, a4paper]{book}
\usepackage[T1]{fontenc} % Use 8-bit encoding that has 256 glyphs
\usepackage[utf8]{inputenc}
\usepackage{fourier} % Use the Adobe Utopia font for the document - comment this line to return to the LaTeX default
\usepackage{listings} % para insertar código con formato similar al editor
\usepackage[spanish, es-tabla]{babel} % Selecciona el español para palabras introducidas automáticamente, p.ej. "septiembre" en la fecha y especifica que se use la palabra Tabla en vez de Cuadro
\usepackage{url} % ,href} %para incluir URLs e hipervínculos dentro del texto (aunque hay que instalar href)
\usepackage{graphics,graphicx, float} %para incluir imágenes y colocarlas
\usepackage[gen]{eurosym} %para incluir el símbolo del euro
\usepackage{cite} %para incluir citas del archivo <nombre>.bib
\usepackage{enumerate}
\usepackage{hyperref}
\usepackage{graphicx}
\usepackage{tabularx}
\usepackage{booktabs}
\usepackage[margin=3cm]{geometry}
\usepackage{array}
\usepackage{graphicx}
\usepackage{rotating}
\usepackage{amsmath}
\usepackage{amssymb}
\usepackage{mathrsfs}
\usepackage{listings}
\usepackage{xcolor}  

\lstdefinestyle{mystyle}{
    backgroundcolor=\color{gray!10},
    commentstyle=\color{green!50!black},
    keywordstyle=\color{blue},
    numberstyle=\tiny\color{gray},
    stringstyle=\color{orange},
    basicstyle=\ttfamily\footnotesize,
    breaklines=true,
    numbers=left,
    numbersep=5pt,
    frame=single,
    rulecolor=\color{black},
    tabsize=2,
    showstringspaces=false
}
\lstset{style=mystyle}


\usepackage[table,xcdraw]{xcolor}
\hypersetup{
	colorlinks=true,	% false: boxed links; true: colored links
	linkcolor=black,	% color of internal links
	urlcolor=cyan		% color of external links
}
\renewcommand{\familydefault}{\sfdefault}
\usepackage{fancyhdr} % Custom headers and footers
\pagestyle{fancyplain} % Makes all pages in the document conform to the custom headers and footers
\fancyhead[L]{} % Empty left header
\fancyhead[C]{} % Empty center header
\fancyhead[R]{Benigno} % My name
\fancyfoot[L]{} % Empty left footer
\fancyfoot[C]{} % Empty center footer
\fancyfoot[R]{\thepage} % Page numbering for right footer
%\renewcommand{\headrulewidth}{0pt} % Remove header underlines
\renewcommand{\footrulewidth}{0pt} % Remove footer underlines
\setlength{\headheight}{13.6pt} % Customize the height of the header

\usepackage{titlesec, blindtext, color}
\definecolor{gray75}{gray}{0.75}
\newcommand{\hsp}{\hspace{20pt}}
\titleformat{\chapter}[hang]{\Huge\bfseries}{\thechapter\hsp\textcolor{gray75}{|}\hsp}{0pt}{\Huge\bfseries}
\setcounter{secnumdepth}{4}
\usepackage[Lenny]{fncychap}


\begin{document}

	% Plantilla portada UGR
	\begin{titlepage}
 
 
\newlength{\centeroffset}
\setlength{\centeroffset}{-0.5\oddsidemargin}
\addtolength{\centeroffset}{0.5\evensidemargin}
\thispagestyle{empty}

\noindent\hspace*{\centeroffset}\begin{minipage}{\textwidth}

\centering
\includegraphics[width=0.9\textwidth]{imagenes/logo_ugr.jpg}\\[1.4cm]

\textsc{ \Large TRABAJO FIN DE GRADO\\[0.2cm]}
\textsc{ GRADO EN INGENIERIA INFORMATICA}\\[1cm]

{\huge\bfseries Análisis de desempeño y predicción de resultados en fútbol mediante mapas de calor \\}
\noindent\rule[-1ex]{\textwidth}{3pt}\\[3.5ex]
\end{minipage}

\vspace{2.5cm}
\noindent\hspace*{\centeroffset}
\begin{minipage}{\textwidth}
\centering

\textbf{Autor}\\ {Benigno Joaquín Parra Campos}\\[2.5ex]
\textbf{Director}\\ {Javier Medina Quero, Juan Luis Jiménez Laredo}\\[2cm]
\includegraphics[width=0.3\textwidth]{imagenes/etsiit_logo.png}\\[0.1cm]
\textsc{Escuela Técnica Superior de Ingenierías Informática y de Telecomunicación}\\
\textsc{---}\\
Granada, Junio de 2025
\end{minipage}
%\addtolength{\textwidth}{\centeroffset}
%\vspace{\stretch{2}}
\end{titlepage}



	% Plantilla prefacio UGR
	\thispagestyle{empty}

\begin{center}
{\large\bfseries Análisis de desempeño y predicción
de resultados en fútbol mediante
mapas de calor }\\
\end{center}
\begin{center}
Benigno\\
\end{center}

%\vspace{0.7cm}

\vspace{0.5cm}
\noindent\textbf{Palabras clave}: \textit{predicción, fútbol, mapa de calor, estadísticas, futbolistas, rendimiento futbolista, machine learning}
\vspace{0.7cm}

\noindent\textbf{Resumen}\\
Este TFG tiene como objetivo la predicción de partidos de fútbol, es decir, determinar quién gana, empata o pierde un partido, así como mostrar el rendimiento de cada jugador en ese partido mediante machine learning. Todo ello con el objetivo de intentar maximizar el rendimiento de un equipo de fútbol a través de los conocimientos otorgados al entrenador sobre su equipo. Para determinar quién gana el partido se utilizan las estadísticas de los jugadores almacenadas en la base de datos, y su mapa de calor, que indica en qué zonas del campo ha estado, posteriormente se define el ganador del partido haciendo uso de algoritmos genéticos. Después, se comprueba la seguridad del resultado con otra base de datos donde están almacenados más de 200 partidos. Por último, se consigue el rendimiento de cada jugador comparando sus estadísticas individuales con las del rival, dando así un mapa de calor resultante con su influencia en el juego.

\cleardoublepage

\begin{center}
	{\large\bfseries Analysis and prediction of performance.
of results in football through
heat maps}\\
\end{center}
\begin{center}
	Benigno\\
\end{center}
\vspace{0.5cm}
\noindent\textbf{Keywords}: \textit{prediction, football, heatmap, statistics, football players, player performance, machine learning}
\vspace{0.7cm}

\noindent\textbf{Abstract}\\
This Bachelor’s Thesis aims to predict football matches, determine who wins, draws, or loses, and display the performance of each player in the match through machine learning. All of this is done with the goal of maximizing a football team's performance by providing the coach with insights about their team. To determine the outcome of a match, player statistics stored in the database are used, along with their heatmaps, which indicate the areas of the field they have occupied. The winner of the match is then defined using genetic algorithms. Afterwards, the reliability of the result is checked against another database containing data from more than 200 matches. Finally, each player's performance is evaluated by comparing their individual statistics with those of their opponent, resulting in a heat map showing their influence on the game.

\cleardoublepage

\thispagestyle{empty}

\noindent\rule[-1ex]{\textwidth}{2pt}\\[4.5ex]

D. \textbf{Tutora/e(s)}, Profesor(a) del ...

\vspace{0.5cm}

\textbf{Informo:}

\vspace{0.5cm}

Que el presente trabajo, titulado \textit{\textbf{Chief}},
ha sido realizado bajo mi supervisión por \textbf{Estudiante}, y autorizo la defensa de dicho trabajo ante el tribunal
que corresponda.

\vspace{0.5cm}

Y para que conste, expiden y firman el presente informe en Granada a Junio de 2025.

\vspace{1cm}

\textbf{El/la director(a)/es: }

\vspace{5cm}

\noindent \textbf{(nombre completo tutor/a/es)}

\chapter*{Agradecimientos}






	% Índice de contenidos
	\newpage
	\tableofcontents

	% Índice de imágenes y tablas
	\newpage
	\listoffigures

	% Si hay suficientes se incluirá dicho índice
	\listoftables 
	\newpage

	% Introducción 
	\chapter{Introducción}

El análisis de datos deportivos ha alcanzado una relevancia creciente en el ámbito del fútbol, donde la complejidad del juego y la gran cantidad de variables involucradas generan un alto grado de incertidumbre en los resultados. Este Trabajo de Fin de Grado (TFG) se centra en abordar dicha incertidumbre mediante el desarrollo de un modelo analítico que permita predecir, de manera post-mortem, el resultado de partidos ya disputados, así como evaluar el rendimiento individual de los jugadores.

A diferencia de las aplicaciones orientadas a predecir resultados futuros, el presente trabajo se basa en el estudio de encuentros ya celebrados con el fin de ofrecer al cuerpo técnico información retrospectiva que facilite el análisis táctico y estratégico. De esta forma, se combinan estadísticas individuales y mapas de calor de cada futbolista para construir una métrica global que refleje la ventaja competitiva de un equipo frente al adversario.

Asimismo, este proyecto también puede resultar de interés para personas ajenas al ámbito profesional del fútbol, como los aficionados que desean anticipar el resultado de un encuentro por mera curiosidad, o incluso aquellos que participan en apuestas deportivas y buscan aumentar sus posibilidades de acierto mediante el análisis estadístico.

\section{Motivación y contexto}

En la actualidad, el fútbol posee una notable relevancia social, con capacidad incluso de influir en el estado emocional de las personas. Según Janhub \cite{impact-football-mood}, las victorias o derrotas de un equipo inciden significativamente en el ánimo de sus seguidores, especialmente cuando asisten al estadio, generando una mayor felicidad en caso de triunfos inesperados. Esta carga emocional incrementa la presión sobre jugadores y entrenadores, dado que las expectativas son elevadas, independientemente del nivel competitivo. En esta línea, Özsari \textit{et al.} \cite{act-footballist} señalan que el rendimiento individual de un futbolista repercute directamente en sus relaciones interpersonales y en su bienestar cotidiano: cuanto mejor actúe en el terreno de juego, mayor será su satisfacción personal.

Otro aspecto a tener en cuenta en el contexto del fútbol es su impacto económico. Tal como expone Aygün \textit{et al.} en el estudio \cite{economy-football}, existe una relación directa entre la evolución del fútbol y el desarrollo económico, especialmente a nivel local. Ambos factores presentan un crecimiento correlativo y se influyen mutuamente, lo que pone de manifiesto que el fútbol no solo tiene una función social destacada, sino que también actúa como motor económico en muchas comunidades.

En este sentido, el fútbol puede ejercer una influencia considerable en la vida de muchas personas. Como expresó el exfutbolista Jorge Valdano, 'el fútbol es la cosa más importante entre las menos importantes', reflejando la paradoja entre su aparente trivialidad y su profundo impacto social. En un contexto donde este deporte adquiere tal relevancia, los equipos buscan optimizar al máximo sus resultados, ya sea a través de estrategias dentro del terreno de juego, como la gestión del tiempo o las protestas al árbitro, o mediante el análisis detallado de datos propios y del rival. En un deporte tan competitivo, incluso el más mínimo detalle puede marcar la diferencia entre la victoria y la derrota.

Si bien el fútbol comprende diversas modalidades, como partidos entre selecciones o clubes, así como encuentros amistosos u oficiales, este estudio se centrará exclusivamente en partidos oficiales disputados por clubes. Por tanto, toda referencia a un 'partido' se referirá a encuentros correspondientes a ligas domésticas, pudiendo ser tanto en categoría masculina como femenina. En lo que respecta a los mapas de calor, se entenderán como representaciones gráficas del terreno de juego que muestran las zonas en las que un jugador ha intervenido con mayor frecuencia. Las áreas de mayor intensidad indican una concentración más alta de acciones, lo que permite identificar las regiones del campo donde cada futbolista tiene una presencia más destacada.

En el contexto de la creciente demanda por obtener resultados deportivos positivos, ha emergido una línea de investigación prometedora: la predicción de partidos mediante técnicas de \textit{machine learning}. Esta disciplina permite analizar grandes volúmenes de datos, como los mapas de calor de los jugadores, para identificar patrones de comportamiento sobre el terreno de juego. Esta información puede ser de gran utilidad para los entrenadores a la hora de tomar decisiones estratégicas, como la elección de la alineación más adecuada en función de las zonas del campo donde cada jugador tiene mayor impacto.

A pesar del potencial de esta metodología, los estudios que combinan mapas de calor y \textit{machine learning} en el ámbito del fútbol son todavía escasos, lo que convierte a esta área en un campo de investigación emergente. Cuanto mayor sea el volumen y la calidad de los datos disponibles, más precisas podrán ser las predicciones generadas por los modelos. Por ello, una mejora futura relevante sería la ampliación de la base de datos utilizada, incorporando más encuentros y variables que permitan afinar los resultados.

No obstante, es importante considerar la naturaleza impredecible del fútbol. Factores aleatorios o situaciones puntuales pueden alterar el desarrollo esperado de un partido, limitando así la capacidad de predicción. Aunque los modelos estadísticos y de \textit{machine learning} pueden estimar probabilidades con cierto grado de precisión, siempre existirá un margen de incertidumbre. Aun así, optimizando estos algoritmos, es posible extraer información valiosa tanto sobre el rendimiento individual como colectivo, lo que podría facilitar la labor del cuerpo técnico en la toma de decisiones orientadas a maximizar las posibilidades de victoria.

\section{Objetivos}
El objetivo principal de este TFG es la predicción de partidos de fútbol a través de las estadísticas de los futbolistas y de las zonas del campo en las que han estado mediante \textit{machine learning}.

Para conseguir este objetivo principal se ha realizado el cumplimiento de los siguientes objetivos específicos:
\begin{itemize}

  \item OE1: Diseñar una métrica global que combine estadísticas individuales y distribución espacial de la actividad de los jugadores para cuantificar la ventaja competitiva de un equipo.
  \item OE2: Optimizar los parámetros de dicha métrica mediante técnicas de \textit{machine learning}.
  \item OE3: Validar la fiabilidad y robustez del modelo con una base de datos independiente de más de 200 partidos ya disputados.
  \item OE4: Generar mapas de calor comparativos que permitan evaluar la influencia individual de los futbolistas y facilitar el análisis táctico.
  \item OE5: Proponer directrices prácticas para el cuerpo técnico basadas en los resultados del modelo, orientadas a mejorar la estrategia de juego y la selección de alineaciones.
    
\end{itemize}

\section{Estructura de la memoria}
Este Trabajo de Fin de Grado (TFG) se organiza en 6 capítulos que parten desde la relevancia social del fútbol hasta la implementación de modelos de \textit{machine learning} basados en mapas de calor. La estructura es la siguiente:

\begin{itemize}
    \item Introducción: En este apartado se contextualiza el impacto del fútbol en la sociedad actual, destacando la creciente importancia de su análisis y predicción. Asimismo, se exponen los objetivos principales que se pretenden alcanzar a lo largo del proyecto.
    \item Estado del arte: Se presenta una revisión bibliográfica sobre los estudios previos relacionados con el fútbol en distintas áreas, tanto desde el punto de vista técnico como analítico. A partir de este análisis, se justifica la línea de desarrollo adoptada en el proyecto, identificando oportunidades de mejora y enfoques poco explorados.
    \item Planificación, metodología y presupuesto del proyecto: Se expone la metodología ágil SCRUM empleada para gestionar el desarrollo del proyecto, detallando su división en iteraciones. Además, se presenta la planificación temporal y se incluye una estimación del presupuesto necesario, considerando los recursos utilizados y el coste asociado al desarrollo.
    \item Análisis del problema y diseño de la solución: En este capítulo se define el problema a abordar y se describe el enfoque adoptado para resolverlo. Se incluyen las formulaciones matemáticas necesarias, así como el diseño general del sistema propuesto.
    \item Implementación: Se describe el proceso de desarrollo de la solución propuesta, detallando las decisiones técnicas, las herramientas utilizadas y la forma en que se integraron los distintos componentes del sistema.
    \item Análisis de resultados: Se presentan y analizan los resultados obtenidos tras aplicar la solución desarrollada, evaluando su rendimiento mediante métricas específicas y discutiendo su validez en el contexto del problema abordado.
    \item Conclusiones y trabajos futuros: Finalmente, se presentan las principales conclusiones del trabajo y se proponen posibles líneas de investigación futura que podrían complementar o ampliar los resultados obtenidos.
\end{itemize}

\section{Licencia de código abierto}
El código desarrollado en el marco de este trabajo se encuentra disponible en el siguiente repositorio público de GitHub: \url{https://github.com/benipr14/TFG/tree/master} Este código ha sido liberado bajo los términos de la licencia MIT, permitiendo su uso, modificación y distribución, siempre que se incluya el aviso de copyright correspondiente y una copia de la licencia.

No obstante, todos los datos empleados en este trabajo han sido obtenidos de la página web de Sofascore \cite{Sofascore}. Cabe señalar que dichos datos están protegidos por derechos de autor y no han sido liberados para su uso público ni académico.


	% Estado del arte
	% 	1. Crítica al estado del arte
	% 	2. Propuesta
	\chapter{Estado del arte}

El fútbol es un deporte altamente impredecible, en el que incluso los detalles más pequeños pueden tener un impacto significativo en el resultado de un partido. Dada la complejidad del juego y la cantidad de variables implicadas, resulta esencial identificar qué áreas han sido objeto de estudio mediante técnicas de aprendizaje automático y cuáles presentan aún oportunidades para futuras investigaciones. Al finalizar este trabajo, se contará con una visión general y estructurada sobre los aspectos del fútbol que han sido previamente explorados, así como aquellos que permanecen relativamente inexplorados. Para la recopilación de información, se han utilizado filtros de búsqueda aplicados a términos relacionados con 'football' o 'soccer', combinados con 'machine learning' o 'deep learning'.

Como criterios de inclusión y exclusión se han usado los siguientes:
\begin{itemize}
    \item Individuos: Los participantes analizados deben ser exclusivamente jugadores de fútbol. No obstante, también se aceptan estudios que incluyan múltiples disciplinas deportivas, siempre que se puedan extraer y utilizar únicamente los datos correspondientes al fútbol.
    \item Técnica: Los datos han tenido que ser procesados mediante \textit{machine learning} o \textit{deep learning} y no mediante otras técnicas.
    \item Predicción: La predicción debe centrarse en el rendimiento individual o colectivo de los jugadores, excluyendo cualquier otro ámbito ajeno al mundo del fútbol, como aspectos económicos o políticos.
    \item Idioma: los textos son escritos originalmente en inglés y no en otro idioma, o hechos basados en otros artículos originales.
\end{itemize}

Como puede observarse, los criterios de inclusión y exclusión han sido definidos con el objetivo de centrarse exclusivamente en el ámbito del fútbol y en los datos relativos a los futbolistas, al mismo tiempo que se asegura la selección de estudios de calidad.

\section{Revisión de la literatura}

En el ámbito del fútbol, la investigación basada en datos se ha centrado principalmente en tres grandes áreas: la predicción de lesiones en los futbolistas, la estimación de los resultados de los partidos y el pronóstico del desarrollo de la carrera deportiva de jugadores individuales. Estas categorías agrupan la mayoría de los estudios recientes orientados al análisis y la predicción dentro de este deporte.

Cada una de estas categorías abarca a su vez diferentes enfoques y subdivisiones específicas, las cuales se irán desarrollando a lo largo de esta sección con el objetivo de ofrecer una visión general del estado actual de la investigación en este ámbito.

El primer tema a tratar es el de las lesiones en futbolistas. Diversos estudios han intentado predecir su aparición, identificando los factores más comunes con el objetivo de prevenirlas. Una de las lesiones más frecuentes es la distensión de isquiotibiales. En el estudio realizado por Ayala et al. \cite{first-injury}, se analizaron 96 jugadores profesionales pertenecientes a 18 equipos distintos, teniendo en cuenta variables como la pierna dominante, el estado psicológico del jugador y otros atributos físicos medibles, como la fuerza o la flexibilidad. Utilizando un algoritmo de \textit{machine learning} AD Tree, llegaron a la conclusión de que, con más de un 78\% de probabilidad, sí había factores que eran predominantes en las 18 lesiones que detectaron, entre ellos la calidad del sueño, el historial de lesiones de isquiotibial o el ángulo de su rodilla al golpear el balón.

En este otro estudio de López-Valenciano \textit{et al.} \cite{second-injury}, se analizaron a 98 jugadores de 4 equipos diferentes y se tuvieron en cuenta factores idénticos al anterior. Su objetivo era minimizar una lesión de cualquier músculo del cuerpo, y para ello utilizaron 4 árboles de decisión que fueron C4.5, SimpleCart, ADTree y RandomTree. Los resultados obtenidos indican que, con un 66\% de probabilidad, podrían ayudar a los entrenadores en la reducción de las lesiones de sus jugadores mediante la modificación de los entrenamientos al detectar posibles riesgos en algunos ejercicios determinados. En este estudio de Rommers \textit{et al.} \cite{third-injury}, se analizaron a 734 niños de entre 10 y 15 años durante una temporada entera. Su objetivo era predecir las lesiones por sobrecarga en los músculos, por lo que se tomaron medidas como la altura, peso, velocidad, resistencia, etc. Utilizaron un algoritmo XGBoost y con una precisión entre el 75\% y 85\% pudieron predecir qué jugadores eran los que tenían más riesgo de lesión.

Como se puede apreciar, en el ámbito de las lesiones se han hecho varios estudios, y la mayoría de ellos son exitosos, lo que nos lleva a pensar que es un campo ya explorado, pero a su vez nos indica que hay campos dentro del fútbol que sí se pueden predecir con precisión.

Vamos ahora a analizar aquellos estudios relacionados con la predicción de partidos o de eventos que ocurran dentro de partidos. En este primer estudio de Dick y Brefeld \cite{first-outcome}, se analizaron 5 partidos de ligas de fútbol profesional con gran variedad de ataques, los cuales 380 fueron exitosos y 715 no lo fueron. Su objetivo era predecir qué situaciones del juego son las más favorables para tener un ataque exitoso. Un ataque exitoso se entiende como aquel en el que el balón está a menos de 25 metros de la portería rival. Para ello, utilizan algunos atributos como la posesión del equipo, las posiciones de cada uno de los jugadores con coordenadas X e Y y el tiempo de juego efectivo, utilizando el algoritmo de optimización de Adam (Adaptive Moment Estimation). Finalmente, llegaron a la conclusión de que, aparte de saber qué ataques podrían ser exitosos, también podría utilizarse para evaluar la posición de los jugadores y la velocidad de los ataques. En este otro estudio mucho más grande de Goes \textit{et al.} \cite{second-outcome}, se analizaron a 26 equipos profesionales durante 4 temporadas y contabilizaron 1.237 ataques exitosos y 11.187 que no lo fueron. Mediante un sistema de video semiautomático, analizaron el centro geométrico entre las distintas líneas del equipo, sobre todo del portero con la defensa y la defensa con los centrocampistas. Se utilizó un algoritmo de k-medias, y se llegó a la conclusión de que, aplicando el centro geométrico a las líneas del equipo en vez de al equipo entero en su totalidad, es más preciso a la hora de predecir ataques exitosos, así como que los ataques exitosos dependen de que los defensores creen espacio para los atacantes y de que estén sincronizados con los centrocampistas.

En el estudio presentado por AlMulla \textit{et al.} en \cite{third-outcome}, se analizaron partidos de la liga catarí disputados entre 2011 y 2022, utilizando datos históricos de los jugadores que participaron en dichos encuentros. Mediante el uso de redes neuronales GRU (Gated Recurrent Units), se logró predecir el equipo ganador con una precisión del 80\%. Además, el análisis se extendió al desempeño de los jugadores en distintas fases del partido, concluyéndose que los defensores desempeñan un papel determinante en el resultado final. Asimismo, se identificó que los intervalos temporales comprendidos entre los minutos 15 y 30 de cada mitad son los más significativos en términos de impacto en el rendimiento general del equipo.

Como se puede observar, estos estudios se centran en el análisis de ataques exitosos y en la predicción del equipo ganador. No obstante, se basan únicamente en estadísticas individuales de los jugadores, sin considerar su relación con las zonas del campo en las que han estado presentes.

Ahora vamos a pasar a analizar 2 estudios sobre los pases de los jugadores. En el primero de ellos, de Szczepańsk y McHale \cite{first-passes}, analizan 760 partidos durante 2 temporadas en la primera división inglesa, teniendo en cuenta cuál es la calidad del jugador que da el pase y su situación antes de darlo. Su objetivo es predecir la dificultad del pase y la probabilidad de que este sea exitoso, y para ello utilizan un algoritmo de naive bayes. El estudio es exitoso y consigue detectar quiénes serían los mejores pasadores. En este otro estudio más preciso de Chawla \textit{et al.} \cite{second-passes}, se analizó al equipo de fútbol 'Arsenal' durante 4 partidos de la primera división inglesa para intentar predecir sus pases. Utilizando un sistema de cámaras y de observadores humanos, contabilizaron 2.932 pases durante el análisis. Para la predicción se tuvieron en cuenta factores como la trayectoria que sigue el jugador en el campo, los toques al balón, pases, tiros y mapas donde dominaba más cada jugador. Para el algoritmo utilizaron un RUSBoost classifier y un multinomial logistic regression y con más de un 90\% de precisión consideraron que sí podían predecir cómo de bueno iba a ser un pase que se iba a dar.

En estos estudios se logró predecir con éxito la calidad y efectividad de los pases, considerando atributos del jugador como su posición en el campo. Sin embargo, dicha técnica no se extendió al ámbito de la predicción del resultado de los partidos.

Después se realizaron otros estudios como el de Link y Hoernig \cite{first-other}, donde se analizaron 60 partidos de la liga alemana, con un total de casi 70.000 acciones captadas mediante un sistema de cámaras semiautomático. Tuvieron en cuenta la posición de los jugadores, posesión del equipo, posesión individual y acciones individuales y colectivas con el balón. Querían determinar cuánto tiempo pasa el balón en la esfera de influencia de un jugador, basándose en la distancia entre los jugadores y el balón, junto con su dirección de movimiento, velocidad y aceleración. Se hace mediante la configuración espacio-temporal del jugador que controla el balón utilizando redes bayesianas. El estudio llega a la conclusión de que podemos saber durante cuánto tiempo ha estado el balón en la esfera de influencia de un futbolista y conocer su influencia en el juego.

En este otro estudio de Montoliu \textit{et al.} \cite{second-other}, se analizaron cuatro equipos de la primera división española durante cuatro temporadas y se tuvieron en cuenta durante los ataques del equipo la posesión del balón, los regates, desmarques, así como los libres directos e indirectos. Su objetivo era predecir cuál sería el patrón de ataque del equipo tanto en jugadas de posesión como a balón parado. Utilizaron un algoritmo de random forest y k-vecinos más cercanos y, entre una precisión entre 67\% y 93\%, podían predecir cuál sería el patrón de ataque del equipo. Este otro estudio de Knauf \textit{et al.} \cite{third-other}, también analiza los patrones de un equipo; para ello, analizó 10 partidos de primera y segunda división alemana durante una temporada, y tuvo en cuenta la trayectoria del jugador en el campo y las oportunidades de gol en el último cuarto del campo. Utilizaron un algoritmo k-medoids y finalmente pudieron predecir si los ataques iban a ser lentos o rápidos, así como la cantidad de pases por ataque.

Predecir patrones de ataque parece ser exitoso utilizando \textit{machine learning}, por lo que nos indica que esta es ya otra área explorada del fútbol.

Por otro lado, existen diversos estudios que analizan la posible trayectoria de un futbolista a lo largo de su carrera profesional. Uno de ellos es el trabajo presentado por Barron \textit{et al.} en \cite{first-career}, en el que se estudia una muestra de 966 jugadores: 209 no profesionales, 637 pertenecientes a la segunda división inglesa y 120 a la primera división inglesa. En este estudio se tienen en cuenta variables como el número de pases, su precisión y consistencia, así como entradas, duelos ganados y tiros. El objetivo principal es predecir el nivel futuro del jugador y la división en la que podría competir en la siguiente temporada, utilizando para ello una red neuronal artificial (ANN). Los resultados muestran una precisión de entre el 69\% y el 77\% en la predicción de la división futura y la trayectoria profesional del jugador a medio plazo.

En otro estudio relevante de Ćwiklinski \textit{et al.} \cite{second-career}, se analizaron 4.700 jugadores pertenecientes a 156 equipos de las ocho principales ligas europeas. Este trabajo considera un total de 29 atributos técnicos, variables psicológicas y datos de rendimiento en partidos anteriores. El objetivo es predecir si la transferencia de un jugador entre dos clubes será exitosa, entendiéndose como tal una mejora en el rendimiento del jugador respecto a su temporada anterior. Para ello, se emplean algoritmos como Random Forest, Naive Bayes y AdaBoost. Aunque los resultados permiten estimar el posible éxito de una transferencia, los autores destacan que estos deben interpretarse más como recomendaciones que como predicciones absolutas.

Otra dimensión interesante de predicción en el fútbol es la elaboración de alineaciones. Aunque existen pocos estudios centrados en este aspecto, uno de ellos, hecho por García-Aliaga \textit{et al.} \cite{starting-up}, analiza más de 50.000 partidos y 30.000 jugadores a lo largo de siete temporadas en 18 ligas nacionales distintas. El objetivo del estudio es determinar cuál sería la posición óptima para cada jugador, teniendo en cuenta sus características individuales, con el fin de maximizar su rendimiento. Para ello, se emplea el algoritmo RIPPER (Repeated Incremental Pruning to Produce Error Reduction), logrando una precisión superior al 73\% en la predicción de la posición más adecuada para cada jugador. Estos resultados pueden resultar especialmente útiles a la hora de diseñar alineaciones más eficaces.

Como se ha podido observar, existen numerosos ámbitos en los que el aprendizaje automático se aplica con éxito en el contexto del fútbol. No obstante, aunque algunos estudios mencionan la posibilidad de generar mapas de calor que reflejen la influencia espacial de cada jugador, ninguno de ellos los utiliza directamente como herramienta para la predicción. Del mismo modo, aunque hay investigaciones que intentan predecir el resultado de un partido a partir de ciertos atributos, no se establece ninguna relación con los mapas de calor que muestran las zonas del campo donde ha intervenido cada jugador.

\section{Contribuciones de este trabajo}
El análisis del estado del arte ha permitido identificar múltiples enfoques aplicados al ámbito del fútbol, como el uso de datos estadísticos, modelos de \textit{machine learning} para la predicción de resultados y representaciones visuales como los mapas de calor. No obstante, se ha detectado una carencia significativa de estudios que combinen de forma sistemática las características individuales de los jugadores con su posicionamiento e influencia espacial durante el partido.

En este contexto, las principales contribuciones de este Trabajo de Fin de Grado son las siguientes:

\begin{itemize}
    \item Integración del rendimiento individual y espacial: Se propone un modelo que relaciona los atributos individuales de los jugadores con su distribución en el terreno de juego, utilizando mapas de calor como fuente principal de información contextual.
    \item Aplicación de técnicas de \textit{machine learning} a datos espaciales específicos: Frente a estudios previos centrados en estadísticas generales o resultados de equipo, este trabajo aplica algoritmos de aprendizaje automático sobre datos derivados directamente de la actividad espacial de los futbolistas, con el objetivo de identificar patrones relevantes en relación con el rendimiento y el resultado del partido.
    \item Exploración de un área poco abordada: Al centrarse en el análisis de mapas de calor como herramienta predictiva, este trabajo abre una línea de investigación emergente que ha sido poco explorada en la literatura existente.
    \item Base para desarrollos futuros: La metodología desarrollada, junto con los resultados obtenidos, sienta las bases para futuras ampliaciones, como el aumento del volumen de datos, la incorporación de nuevas variables contextuales o la aplicación a competiciones y niveles distintos.
\end{itemize}

Estas contribuciones permiten situar este trabajo dentro del campo de la analítica deportiva, proporcionando un enfoque novedoso que combina el análisis espacial y estadístico con técnicas avanzadas de predicción.

	
	\chapter{Planificación}

\section{Metodología utilizada}
Para la organización y desarrollo de este TFG se ha optado por seguir una metodología ágil, concretamente el marco de trabajo SCRUM. Esta elección permite estructurar el proyecto en ciclos iterativos e incrementales, favoreciendo la adaptabilidad, el control del progreso y la mejora continua.

El proyecto se ha dividido en tres iteraciones principales, cada una con una duración determinada y unos objetivos específicos. 

Esta planificación permite mantener una visión clara del desarrollo del trabajo, asegurando que se cumplen los plazos establecidos y que se cubren progresivamente todos los aspectos necesarios del proyecto, desde la recolección y análisis de datos hasta el diseño y validación del modelo propuesto.

En total, se han definido 17 historias de usuario para el proyecto. Cada una de ellas cuenta con una estimación del esfuerzo necesario, basada en la escala de Fibonacci. Además, se les ha asignado un nivel de prioridad, que puede ser: muy alta, alta, media o baja. Las historias están ordenadas según el cronograma previsto de desarrollo, aunque este orden puede modificarse en función de las necesidades del proyecto.

\begin{table}[h]
    \centering
    \begin{tabular}{|>{\centering\arraybackslash}p{1.5cm}|>{\centering\arraybackslash}p{6cm}|>{\centering\arraybackslash}p{2cm}|>{\centering\arraybackslash}p{1.8cm}|}
        \hline
        \textbf{ID} & \textbf{Historia de Usuario} & \textbf{Prioridad} & \textbf{Estimación} \\ \hline
        HU.01 & Reunión inicial con tutores & Muy alta & 2 \\ \hline
        HU.02 & Revisión de la literatura & Alta & 13\\ \hline
        HU.03 & Redacción del estado del arte& Alta & 8\\ \hline
        HU.04 & Definición de las estadísticas a almacenar & Alta & 3 \\ \hline
        HU.05 & Diseño de la base de datos & Alta & 5\\ \hline
        HU.06 & Implementación de programa para recolectar datos & Alta & 8\\ \hline
        HU.07 & Diseño de algoritmo para definir ganador& Alta & 8 \\ \hline
        HU.08 & Implementación algoritmo para definir ganador por pesos& Alta & 5\\ \hline
        HU.09 & Implemetación programa básico sin pesos & Media & 2 \\ \hline
        HU.10 & Análisis de las variables del agoritmo genético & Media & 5 \\ \hline
        HU.11 & Implementación del algoritmo genético por pesos & Alta & 5 \\ \hline
        HU.12 & Implementación del algoritmo genético por pesos y delta & Alta & 5 \\ \hline
        HU.13 & Análisis de los datos recogidas por los algoritmos& Alta & 8\\ \hline
        HU.14 & Diseño algoritmo para obtención de datos individuales del partido & Baja & 3 \\ \hline
        HU.15 & Implemetación del algoritmo para obtención de datos individuales del partido& Baja & 5 \\ \hline
        HU.16 & Análisis de los datos de cada jugador individual& Baja & 5 \\ \hline
        HU.17 & Redacción de la memoria & Alta & 8 \\ \hline
    \end{tabular}
    \caption{Listado inicial de tareas}
    \label{tab:placeholder_label}
\end{table}

La manera en la que se han dividido las historias en las distintas iteraciones ha sido la siguiente:
\begin{itemize}
    \item Iteración 1: HU.01, HU.02, HU.03, HU.04, HU.05 (31 puntos de historia)
    \item Iteración 2: HU.06, HU.07, HU.08, HU.09, HU.10, HU.11 (33 puntos de historia)
    \item Iteración 3: HU.12, HU.13, HU.14, HU.15, HU.16, HU.17 (35 puntos de historia)
    
\end{itemize}

Las historias se han dividido para intentar que haya un número similar de puntos de historia en cada iteración. 

La primera iteración, que es la que menos tiene, sirve también para inicializarme en el proyecto con la reunión inicial con tutores, la revisión del estado del  arte y su redacción, la decisión de las estadísticas a almacenar y el diseño que tendría la base de datos.

La segunda iteración se basa principalmente en la implementación de los algoritmos del programa, y la tercera iteración en la implementación del último algoritmo, de su análisis, lo relativo al rendimiento individual de cada jugador y la redacción de la memoria.

Se han intentado dividir las historias para que cada iteración siga una temática parecida y, al mismo tiempo, seguir un orden en la realización del proyecto.

\section{Temporización}
Para la temporización se ha decidido asignarle a cada iteración un tiempo de 4 semanas, empezando el día 1 de marzo y acabando el 31 de mayo, es decir, a cada iteración le pertenece un mes, desde marzo hasta mayo incluido. Dentro de cada iteración se ha dividido el tiempo conveniente para cada historia, haciendo primero aquellas que son necesarias para continuar. El diagrama de Gantt de todo el proyecto se puede ver en la siguiente página.

\begin{figure}[p]
    \centering
    \begin{sideways}
        \includegraphics[width=\textheight, height=\textwidth, keepaspectratio]{plantilla-TFG-ETSIIT/doc/imagenes/Diagrama-Gantt.png}
    \end{sideways}
    \caption{Diagrama de Gantt del proyecto}
    \label{fig:imagen-ajustada}
\end{figure}

\section{Presupuesto}
\subsection*{Coste de personal}
Para este proyecto se necesita la contratación de un informático junior. Según \cite{sueldo-junior}, el sueldo de un programador junior en España es de 13.46 €/hora a lo que hay que
sumarle un aproximado 25\% por seguro médico, IRPF y demás impuestos. Durante este proyecto se ha trabajado durante 300 horas aproximadamente. También se necesita a un ingeniero en sistemas para mantener el servidor, según \cite{sueldo-sistemas} su sueldo medio es de 24.20 €/hora más un 25\% aproximado otra vez por impuestos, y ha trabajado 1 hora al día para verificar que todo estaba bien durante los 2 últimos meses del proyecto. También se va a prever 1 año más para dar soporte al proyecto y pueda mantenerse.

\begin{table}[h!]
\centering
\begin{tabular}{|c|c|c|c|}
\hline
\textbf{Descripción} & \textbf{Horas} & \textbf{€/hora} & \textbf{Coste(€)} \\
\hline
Programador junior & 300 & 13.46 & 4.038 \\
\hline
Ingeniero en sistemas & 425 & 24.20 & 10.285 \\
\hline
Total &  &  & 14.323 \\
\hline
\end{tabular}
\caption{Coste de personal}
\label{tab:ejemplo}
\end{table}

\subsection*{Coste de material}
En cuanto a hardware, se ha hecho de un portátil Lenovo Ideapad Gaming 3 cuyo coste de compra fue de 900 €. Hace 4 años que se compró, pero solo se va a usar durante un año, debido al mantenimiento del servidor. Simulando que se vaya a lanzar este proyecto de manera profesional, también se necesita el uso de un servidor con capacidad para albergar gran cantidad de datos. Se ha utilizado un servidor físico (HPE ProLiant MicroServer Gen10+) como plataforma para alojar la base de datos MongoDB, con un coste estimado de 775.98 €. Esto permite replicar un entorno profesional con servicios persistentes y seguros de gestión de datos.

En cuanto a software, se hace uso de MongoDB, Visual Studio, GitHub y Ubuntu; todos son gratuitos.

\begin{table}[h!]
\centering
\begin{tabular}{|c|c|c|c|c|}
\hline
\textbf{Descripción} & \textbf{Coste(€)} & \textbf{Años} & \textbf{Años usado} & \textbf{Coste(€)} \\
\hline
Portátil & 900 & 5 & 1 & 180 \\
\hline
Servidor & 775.98 & 1 & 1 & 775.98 \\
\hline
MongoDB & 0 & 1 & 1 & 0 \\
\hline
Visual Studio & 0 & 1 & 1 & 0 \\
\hline
GitHub & 0 & 1 & 1 & 0 \\
\hline
Ubuntu & 0 & 1 & 1 & 0 \\
\hline
Total &  & & & 955.98 \\
\hline
\end{tabular}
\caption{Coste de material}
\label{tab:ejemplo}
\end{table}

\subsection*{Costes indirectos}
Para los costes indirectos se estima un 10\% del gasto total del proyecto.
\begin{table}[h!]
\centering
\begin{tabular}{|c|c|c|c|}
\hline
\textbf{Descripción} & \textbf{Total} & \textbf{\%} & \textbf{Coste(€)} \\
\hline
Costes indirectos & 15.278,98 & 10 & 1.527,89 \\
\hline
Total &  &  & 16.806,87 \\
\hline
\end{tabular}
\caption{Costes indirectos}
\label{tab:ejemplo}
\end{table}

El coste total del proyecto sería de 16.806,87 €.

	% Análisis del problema
	% 1. Análisis de requisitos
	% 2. Análisis de las soluciones
	% 3. Solucion propuesta
	% 4. Análisis de seguridad
	\chapter{Análisis del problema y diseño de la solución}

Para este TFG se quiere conseguir una predicción en el mundo del fútbol para poder decir con cierta seguridad el ganador de un partido. Para ello, se ha diseñado un algoritmo que utiliza las estadísticas de los futbolistas, entre ellas su mapa de calor, para predecir el ganador, y después, respaldar ese resultado con una base de datos con los resultados de partidos ya jugados.

\section*{Definición del algoritmo}
Este algoritmo presenta un modelo analítico para predecir el resultado de un partido de fútbol, basado en las estadísticas individuales de los jugadores y de sus mapas de calor. Se separan las estadísticas en categorías constructivas y destructivas, se normalizan y se ponderan según la distribución espacial de cada jugador en el campo, dividiendo el campo en zonas con distintos pesos. El modelo evalúa la ventaja en cada zona mediante una simetría entre zonas ofensivas y defensivas, para finalmente construir un score final que permite definir el ganador del partido. Más adelante, mediante un algoritmo genético, se intentará optimizar los pesos que se le dan a cada zona para intentar obtener el máximo rendimiento posible.

\section{Base de datos}
En el ámbito del análisis deportivo, especialmente en el fútbol, resulta de gran interés predecir el resultado de un partido de fútbol a partir de datos complejos que van más allá del simple marcador. En este proyecto se parte de una base de datos de partidos de la liga española, tanto femenina como masculina, en la que se dispone de tres tipos de información:

\begin{itemize}
    \item Resultado del partido: se almacena quien gana, empata o pierde el partido, así como el local, visitante y la temporada en que se jugó.
    \item Mapas de calor de cada jugador: muestra la distribución espacial de su actividad en el partido.
    \item Estadísticas individuales de cada jugador en el partido: pueden ser constructivas o destructivas.
    
\end{itemize}

Para la implementación del sistema se ha optado por el uso de MongoDB como sistema de gestión de bases de datos no relacional debido a su flexibilidad y facilidad con el tratamiento de los datos. Aunque también se ha hecho uso de MongoDB Compass como interfaz gráfica, debido a que todo es mucho más intuitivo y fácil de mostrar.

Para el almacenamiento de los partidos ya jugados se han almacenado los atributos de local, visitante, temporada y el ganador del partido. No se ha visto necesario almacenar el número de goles, pues no nos interesa saber por cuánto se ha ganado el partido, sino quién lo ha ganado. La descripción de los atributos es la siguiente:

\begin{table}[H]
    \centering
    \begin{tabular}{|c|c|}
        \hline
        \textbf{Atributo} & \textbf{Descripción} \\
        \hline
        Local & Nombre del equipo que jugó el partido como local \\
        \hline
        Visitante & Nombre del equipo que jugó el partido como visitante \\
        \hline
        Temporada & Temporada en la que se jugó el partido \\
        \hline
        Resultado & Indica si la victoria fue local, visitante o hubo empate \\
        \hline
    \end{tabular}
    \caption{Atributos de los partidos de la base de datos}
    \label{tab:ejemplo}
\end{table}

También es importante aclarar que en el fútbol español, para diferenciar a los equipos masculinos y femeninos se les añade un 'F' al final del nombre, por lo que esta será la manera de almacenarlos en la base de datos y de diferenciarlos con los masculinos. Un ejemplo implementado desde MongoDB Compass sería:

\begin{figure}[H]
    \centering
    \includegraphics[width=0.7\textwidth]{plantilla-TFG-ETSIIT/doc/imagenes/BD_partido.png}
    \caption{Entrada de partido en MongoDB Compass}
    \label{fig:etiqueta-imagen}
\end{figure}

Después, para cada partido se almacenan las estadísticas individuales de cada futbolista de ese partido concreto, eso incluye datos básicos para todos como el nombre, equipo, rival, posición, condición de local o visitante y su mapa de calor. También se almacenan otras más concretas que han sido seleccionadas según el criterio de Sofascore \cite{just-estadisticas}, que considera que esas son las más relevantes de los partidos. Los datos básicos de los jugadores son estos:

\begin{table}[H]
    \centering
    \begin{tabular}{|c|c|}
        \hline
        \textbf{Atributo} & \textbf{Descripción} \\
        \hline
        Nombre & Nombre del futbolista al que pertenecen esos datos \\
        \hline
        Posición & Posición en la que el futbolista jugó el partido \\
        \hline
        Equipo & Nombre del equipo en el que el futbolista jugó el partido \\
        \hline
        Rival & Nombre del equipo rival al que se enfrentó en el partido \\
        \hline
        Condición & Condición en que el futbolista jugó, puede ser local o visitante \\
        \hline
        Temporada & Temporada en la que se jugó el partido \\
        \hline
    \end{tabular}
    \caption{Atributos básicos de futbolistas de la base de datos}
    \label{tab:ejemplo}
\end{table}

La posición de los futbolistas se ha representado mediante sus abreviaturas, que son las siguientes:
\begin{itemize}
    \item POR: Portero
    \item LD: Lateral derecho
    \item DFC\_D: Defensa central por la derecha
    \item DFC\_I: Defensa central por la izquierda
    \item LI: Lateral izquierdo
    \item MCD: Medio centro defensivo
    \item MC: Centrocampista
    \item MD: Medio derecho
    \item MI: Medio izquierdo
    \item MCO: Medio centro ofensivo
    \item DC: Delantero centro
    \item ED: Extremo derecho
    \item EI: Extremo izquierdo
\end{itemize}

Y las estadísticas específicas que se almacenan son las siguientes:

\begin{table}[H]
    \centering
    \begin{tabular}{|c|c|}
        \hline
        \textbf{Atributo} & \textbf{Descripción} \\
        \hline
        Asistencias & Número de asistencias que dio el futbolista en el partido \\
        \hline
        Tiros & Número de tiros a puerta que hizo el futbolista en el partido \\
        \hline
        Pases & Número de pases exitosos que dio el futbolista en el partido \\
        \hline
        Pases clave & Número de pases clave exitosos que dio el futbolista en el partido \\
        \hline
        Regates & Número de regates exitosos que hizo el futbolista en el partido \\
        \hline
        Duelos ganados & Número de duelos que ganó el futbolista en el partido \\
        \hline
        Acciones defensivas & Número de acciones defensivas que hizo el futbolista en el partido \\
        \hline
        Despejes & Número de despejes que dio el futbolista en el partido \\
        \hline
        Recuperaciones & Número de recuperaciones que hizo el futbolista en el partido \\
        \hline
        Entradas exitosas & Número de entradas exitosas que hizo el futbolista en el partido \\
        \hline
        Paradas & Número de paradas que hizo el futbolista en el partido (solo para porteros)\\
        \hline
        Mapa de calor & Representación espacial que muestra donde tocó el futbolista el balón \\
        \hline
    \end{tabular}
    \caption{Estadísticas de futbolistas de la base de datos}
    \label{tab:ejemplo}
\end{table}

Un ejemplo implementado desde MongoDB Compass sería:
\begin{figure}[H]
    \centering
    \includegraphics[width=0.7\textwidth]{plantilla-TFG-ETSIIT/doc/imagenes/BD_futbolista.png}
    \caption{Ejemplo almacenamiento de futbolista en BD}
    \label{fig:etiqueta-imagen}
\end{figure}

Para el mapa de calor se almacena como una matriz donde cada entrada tiene 2 coordenadas x e y,  que indican dónde tocó el balón el futbolista dentro del terreno de juego.

\begin{figure}[H]
    \centering
    \includegraphics[width=0.7\textwidth]{plantilla-TFG-ETSIIT/doc/imagenes/BD_mapa_calor.png}
    \caption{Ejemplo almacenamiento de mapa de calor en BD}
    \label{fig:etiqueta-imagen}
\end{figure}

Todos los datos empleados en este trabajo han sido obtenidos de la página web de Sofascore \cite{Sofascore}. Cabe señalar que dichos datos están protegidos por derechos de autor y no han sido liberados para su uso público ni académico.

\section{Planteamiento del problema}
El objetivo es desarrollar una expresión analítica basada en sumatorios y ponderaciones que, usando únicamente las estadísticas normalizadas y los mapas de calor, determine el resultado del partido. La idea central es que la aportación de cada equipo se obtenga a partir de la suma de las contribuciones de sus jugadores en distintas zonas del campo, comparándose de forma simétrica las zonas de ataque de un equipo con las zonas defensivas del rival. Para ello, se van a seguir los siguientes pasos:

\begin{enumerate}
    \item \textbf{Separación de las estadísticas por naturaleza ofensiva y defensiva}: Cada jugador posee acciones que favorecen la creacion de juego (constructivas) y acciones que impiden el avance del rival (destructivas). Las acciones consideradas constructivas son: asistencias, tiros, pases, pases clave, regates y duelos ganados. Las acciones defensivas son: acciones defensivas, despejes, recuperaciones, entradas exitosas y paradas.
    \item \textbf{Normalización de las estadísticas}: Todas las estadísticas se normalizan a valores entre 0 y 1, donde 0 indica que la acción no se realizó durante el partido y 1 indica que el jugador fue el máximo ejecutor de esa acción.
    \item \textbf{Impacto espacial a través de heatmaps}: cada jugador tiene un mapa de calor. Se define:
    \begin{itemize}
        \item $ A_i: \text{ área total (número de píxeles) del heatmap del jugador } i.$
        \item $ a_{i,j}: \text{ número de píxeles del jugador } i \text{ que caen en la zona } j. $
        \item $ p_{i,j} = \frac{a_{i,j}}{A_i}; \text{ impacto relativo del jugador } i \text{ en la zona } j. $
    \end{itemize} 

    \item \textbf{División del terreno en zonas:} El campo se divide en un número fijo de 24 zonas asignando a cada una un peso $w_j$ que refleja su importancia. Es importante 
    considerar la simetría ya que la zona que es defensiva para un equipo es ofensiva para el otro (por ejemplo, la portería de uno es el área de ataque del contrario).
    \item \textbf{Predicción del resultado:} A partir de la aportación combinada de todas las zonas y de ambos equipos, se establece un score global que, según su signo (y magnitud),
    predice la victoria del equipo local, la victoria del visitante o un empate.

\end{enumerate}

\section{Desarrollo del modelo matemático}
\subsection*{Separación y normalización de estadísticas}

Cada jugador $i$ registra diversas estadísticas durante el partido, que se dividen en dos
categorías:

\begin{itemize}
    \item \textbf{Juego constructivo (ofensivo):} Conjunto $C$ de acciones que favorecen la creación de juego.
    \item \textbf{Juego destructivo (defensivo):} Conjunto $D$ de acciones que dificultan el avance del rival.
\end{itemize}

Sea $f_{i,k}$ la estadística cruda $k$ del jugador $i$. Para comparar valores entre jugadores y
partidos, cada estadística se normaliza a un valor en el intervalo [0, 1]. Una forma de
normalizar es:
\[
f_{i,k}^{\text{norm}} = \frac{f_{i,k}}{\max\{f_{j,k} : j \text{ en el partido}\}}
\]
donde máx\{$f_{i,k}$\} es el máximo observado de la estadística $k$ en el partido.
Posteriormente, se calculan dos scores diferenciados:
\[
S_i^c = \sum_{k \in C} f_{i,k}^{\text{norm}} \quad y \quad S_i^d = \sum_{k \in D} f_{i,k}^{\text{norm}}
\]

\section{Calculo del impacto espacial mediante el mapa de calor}
Cada jugador dispone de un mapa de calor, que es una representación en píxeles de su
actividad en el terreno durante el partido.
\subsection*{Área total de acción}
Se define el área total de acción del jugador $i$ como:
\[
A_i = \sum_{j=1}^{N} a_{i,j}
\]

donde $a_{i,j}$ es el número de píxeles del heatmap que caen en la zona $j$ y $N$ es el número total de zonas en que se ha dividido el campo, en este caso 24. Esta suma representa la
totalidad de la presencia del jugador en el terreno.

\subsection*{Impacto relativo en cada zona}
Cada zona $j$ del campo se considera una subdivisión del mapa de calor. El impacto o la
proporción de la acción del jugador en la zona $j$ se define como:
\[
p_{i,j} = \frac{a_{i,j}}{A_i}
\]

Este valor indica la fracción del área total de acción del jugador que se concentra en la
zona $j$.

\section{Cáculo de la aportación individual a cada zona}
Se pondera la calidad (score) del jugador por su presencia en cada zona. Así, para
cada jugador $i$ y zona $j$ se definen dos aportaciones:

\begin{itemize}
    \item \textbf{Aportación constructiva:} 
    \[ C^c_{i,j,ofensivo} = p_{i,j} \cdot S_i^c \quad \]
    \item \textbf{Aportación destructiva:}
    \[ \quad C^d_{i,j, defensivo} = p_{i,j} \cdot S_i^d \]
\end{itemize}

De esta forma, se distribuye el desempeño global del jugador en función de la intensidad
de su presencia en distintas áreas del campo.

\section{Aportación total del equipo en cada zona}
Para cada equipo $T$ (local o visitante), se agregan las contribuciones de todos sus
jugadores en cada zona $j$:

\[ C^c_{T,j} = \sum_{i \in T} C^c_{i,j} \quad , \quad C^d_{T,j} = \sum_{i \in T} C^d_{i,j} \]

Esto permite obtener, para cada zona, un indicador global de la capacidad ofensiva y
defensiva del equipo, considerando la distribución espacial de la actividad.

\section{Evaluación de la ventaja zonal y cálculo del score global}
\subsubsection*{Evaluación de las zonas de ataque y defensa}

Para evaluar la ventaja de un equipo se comparan las aportaciones en las zonas correspondientes:

\begin{itemize}
    \item \textbf{Para cada zona del campo} : Se compara la aportación constructiva local con la aportación defensiva del equipo visitante y viceversa:
    \[ \Delta_j = w_{ofensivo} \cdot C^c_{\text{local},j} - w_{defensivo} \cdot C^d_{\text{visitante},m(j)} \]
    
    \[ \Delta_k' = w_{defensivo} \cdot C^c_{\text{visitante},k} - w_{ofensivo} \cdot C^d_{\text{local},m(k)} \]
    
\end{itemize}

Aquí, $w_{ofensivo}$ y $w_{defensivo}$ son los pesos asignados a cada zona, en función de si son estadísticas ofensivas o defensivas (valores en el intervalo [0, 1]) que
reflejan la importancia relativa de cada área del campo.

\subsubsection*{Score global del partido}
Se calcula el score total de cada equipo y se comparan:

\[S_1 = \sum_{j=1}^{N} \Delta_j\] 

\[S_2 = \sum_{k=1}^{N} \Delta_k \]

El score final es la diferencia entre ambos:

\[ S = S_1 - S_2\]

\subsubsection*{Regla de decisión}
La predicción del resultado del partido se basa en el valor de S:
\begin{itemize}
    \item Si $S > \delta$, se predice victoria del equipo local.
    \item Si $S < -\delta$, se predice victoria del visitante.
    \item Si $|S| \leq \delta$, se predice empate.
\end{itemize}

El parámetro $\delta$ es un umbral pequeño que se ajusta para tener en cuenta variaciones o incertidumbres en la estimación; en este caso, se ha decidido, tras varias pruebas hechas a mano, que el valor sea 2.

\section{Algoritmo genético}
En el contexto actual de creciente complejidad en la resolución de problemas computacionales, los algoritmos evolutivos han ganado relevancia como herramientas eficaces para abordar tareas que presentan un elevado grado de dificultad, especialmente aquellas que no pueden resolverse de forma óptima mediante métodos deterministas tradicionales. Entre estos algoritmos, los algoritmos genéticos destacan por su capacidad para encontrar soluciones aproximadas de alta calidad en espacios de búsqueda amplios.

Inspirados en los principios de la evolución natural propuestos por Charles Darwin, los algoritmos genéticos simulan procesos como la selección natural, el cruce genético y la mutación para evolucionar una población de posibles soluciones hacia un óptimo. Gracias a su carácter exploratorio y adaptativo, los algoritmos genéticos han sido aplicados con éxito en diversas áreas, como la optimización de funciones, el diseño de redes neuronales, la planificación de rutas o la programación automática.

El objetivo del algoritmo genético aplicado a mi proyecto es tratar de encontrar unos vectores de pesos de ataque y defensa adecuados para ponderar las distintas zonas del campo, ya que hasta ahora se habían elegido de manera manual, aunque con cierta lógica.

Para implementarlo, los individuos de la población estarían formados por los vectores de ataque y defensa unidos en uno solo, y el objetivo sería maximizar según la función de evaluación que se basa en ejecutar el vector en todos los partidos de la base de datos, dando un porcentaje de acierto del vector. Para ello se utilizan distintos atributos de población, tasa de mutación y número de generaciones, y se prueban las combinaciones de todos para quedarnos con la mejor. Las especificaciones de los atributos y su implementación se pueden ver en la sección 5.4, Algoritmo genético.


\subsection*{Otros algoritmos}
Se ha implementado también el mismo algoritmo genético, pero con $\delta$, es decir, se puede predecir el vector de pesos y $\delta$, que es el valor que permite definir a partir de dónde se decide el ganador del partido, que hasta ahora había sido puesto a 2 de manera manual. Las especificaciones de los atributos y su implementación se pueden ver en la sección 5.5, Algoritmo genético con $\delta$.

Por último, se ha implementado un algoritmo donde no se tiene en cuenta el mapa de calor de los futbolistas con el objetivo de mostrar la comparación de los resultados cuando estos no se usan.
Su implementación se puede consultar en la sección 5.2, Algoritmo sin pesos,




	% Implementación
	\chapter{Implementación}

\section{Datos}
Los datos han sido recolectados de la página oficial de Sofascore \cite{Sofascore} mediante scrapeo. Para ver las estadísticas de un partido desde la página, basta con irse a ese partido y a 'Estadísticas del jugador' donde se mostrarán todos los jugadores que jugaron el partido. El mapa de calor lo obtengo a través de una API que tiene una función predefinida para eso. 

Para poder meter los datos de un partido hay que escribir el nombre del jugador, equipo, rival que se desea almacenar y la url de ese partido y automáticamente se meterán sus datos en la base de datos.

El código son más de cien líneas y la mayoría son ajustes del scrapeo, por lo que se ha considerado no ponerlo aquí, pero se puede ver directamente en GitHub.

\subsection{Mapas de calor}
Las estadísticas de los jugadores se almacenan como enteros o flotantes y muestran simplemente el valor de dicha estadística (por ejemplo, pases clave, 7), sin embargo, los mapas de calor son diferentes. En cada uno se almacena una lista con una pareja de enteros que representan las coordenadas x e y en el campo. 

Para representar el campo de fútbol se ha hecho uso de las librerías matplotlib y mplsoccer para su representación, esta última es la que permite dibujar el campo. El código para representar un campo vacío sería este:

\begin{lstlisting}[language=Python, caption={Representación campo vacío}, label={lst:codigo-python}]
import matplotlib.pyplot as plt
from mplsoccer import Pitch

fig, ax = plt.subplots(figsize=(16, 9))
pitch = Pitch(pitch_type='opta')
pitch.draw(ax=ax)
plt.show()
\end{lstlisting}

La representación del campo vacío quedaría:

\begin{figure}[H]
    \centering
    \includegraphics[width=0.7\textwidth]{plantilla-TFG-ETSIIT/doc/imagenes/Campo_vacio.png}
    \caption{Reprentación campo de fútbol}
    \label{fig:etiqueta-imagen}
\end{figure}

Es importante recalcar que el punto (0, 0) está abajo a la izquierda y el (99, 99) arriba a la derecha, por lo que los mapas siempre se verán de izquierda a derecha, incluso cuando el jugador juegue de visitante se ajustarán sus datos para que se vea de la misma manera siempre y no dejar lugar a dudas.

Para representar el mapa de un jugador concreto solo tenemos que obtenerlo de su base de datos y representarlo de la misma manera que antes, pero rellenando el campo con sus puntos, el código es el siguiente:

\begin{lstlisting}[language=Python, caption={Mapa de calor de un futbolista}, label={lst:codigo-python}]

from pymongo import MongoClient
import pandas as pd
import matplotlib.pyplot as plt
from mplsoccer import Pitch
import warnings
from matplotlib.colors import LinearSegmentedColormap

warnings.filterwarnings("ignore")

# Conexion a mongoDB
cliente = MongoClient("mongodb://localhost:27017")
db = cliente["TFG"]
coleccion_jugadores = db["jugadores"]

# Buscar un jugador
jugador = coleccion_jugadores.find_one({"nombre": "Luis Milla", "equipo": "Getafe", "rival": "Villareal"})

# Verificar si se encontro el jugador
if jugador:

    mapa_calor_lista = jugador["mapa_calor"]
    mapa_calor_df = pd.DataFrame(mapa_calor_lista)
    colors = [(0, "white"), (0.5, "orange"), (1, "red")]
    custom_cmap = LinearSegmentedColormap.from_list("custom_cmap", colors)

    fig, ax = plt.subplots(figsize=(16, 9))

    pitch = Pitch(pitch_type='opta')
    pitch.draw(ax=ax)
    pitch.kdeplot(mapa_calor_df.x, mapa_calor_df.y, ax=ax,
                fill = True,
                levels=100,
                thresh=0.08,
                zorder=-1,
                bw_adjust=0.15,
                cmap="OrRd")

    # Anadir titulo personalizado
    nombre = jugador["nombre"]
    equipo = jugador["equipo"]
    rival = jugador["rival"]
    plt.title(f"Mapa de calor del jugador \"{nombre}\" del equipo \"{equipo}\", rival \"{rival}\"", fontsize=18)

    plt.show() 

else:
    print("Jugador no encontrado.")

\end{lstlisting}

Los ajustes de los valores de representación del mapa se han hecho a mano dejando los que se consideraban más adecuados para la visualización del mapa de calor. Vamos a ejemplificarlo con algunos jugadores.

El jugador Luis Milla del Getafe es centrocampista y su mapa de calor en el partido contra el Villarreal de liga de la temporada 2024/2025 se ve de la siguiente manera:

\begin{figure}[H]
    \centering
    \includegraphics[width=0.7\textwidth]{plantilla-TFG-ETSIIT/doc/imagenes/Mapa_MC.png}
    \caption{Ejemplo representación de MC}
    \label{fig:etiqueta-imagen}
\end{figure}

Vamos a ver más ejemplos de otras posiciones distintas. El de un lateral derecho se vería algo parecido a esto:

\begin{figure}[H]
    \centering
    \includegraphics[width=0.7\textwidth]{plantilla-TFG-ETSIIT/doc/imagenes/Mapa_LD.png}
    \caption{Ejemplo representación de LD}
    \label{fig:etiqueta-imagen}
\end{figure}

Como se puede apreciar, al ser un lateral derecho tiene predominancia en la parte de abajo del campo ya que recordemos se ve izquierda a derecha y de abajo a arriba.

Un ejemplo de un delantero extremo izquierdo se vería algo así:

\begin{figure}[H]
    \centering
    \includegraphics[width=0.7\textwidth]{plantilla-TFG-ETSIIT/doc/imagenes/Mapa_EI.png}
    \caption{Ejemplo representación de EI}
    \label{fig:etiqueta-imagen}
\end{figure}

Como se ve en la imagen, predomina principalmente la zona de arriba a la derecha que es la que pertenece al extremo izquierdo.

Por último, vamos a ver el mapa de calor de un portero, al que rara vez vamos a ver fuera del área.

\begin{figure}[H]
    \centering
    \includegraphics[width=0.7\textwidth]{plantilla-TFG-ETSIIT/doc/imagenes/Mapa_POR.png}
    \caption{Ejemplo representación de POR}
    \label{fig:etiqueta-imagen}
\end{figure}

También es importante recalcar que los mapas de calor pueden ser algo aleatorios, aunque los jugadores suelen respetar bastante su posición, en los partidos ocurren eventos inesperados y puede que haya influido en otras zonas, aunque son casos aislados. Por ejemplo, en el caso del extremo izquierdo tocó el balón 3 veces en su área, algo que no es habitual por lo que se puede suponer que quizá sea por defender jugadas a balón parado. O el primero que hemos visto, Luis Milla, que es centrocampista toca el balón en las 2 esquinas del campo por lo que se intuye que ha sido para sacar los córneres.

\section{Algoritmo sin zonas}

Una vez obtenidos los datos de los jugadores en la base de datos se va a pasar a explicar la implemetación del algoritmo sin zonas, en decir, solo teniendo en cuenta las estadísticas de los futbolistas y sin dividir el campo en zonas. No se ha considerado poner todas las funciones porque son demasiadas, pero sí la función principal del programa que es la siguiente:

\begin{lstlisting}[language=Python, caption={Algoritmo sin zonas}, label={lst:codigo-python}]
if not JUGADORES_CACHE:
    inicializar_cache_jugadores()
#Bucle que recorre todos los partidos
for partido in coleccion_partidos.find():
    # Obtener jugadores
    jugadores1 = [jug for jug in JUGADORES_CACHE.values() if jug["equipo"] == partido["Local"] and jug["rival"] == partido["Visitante"] and jug["temporada"] == partido["Temporada"]]
    jugadores2 = [jug for jug in JUGADORES_CACHE.values() if jug["equipo"] == partido["Visitante"] and jug["rival"] == partido["Local"] and jug["temporada"] == partido["Temporada"]]
    
    maximos = calcularMaximos(jugadores1, jugadores2)

    suma_total_ofensiva_e1 = 0
    suma_total_defensiva_e1 = 0
    suma_total_ofensiva_e2 = 0
    suma_total_defensiva_e2 = 0

    for j in jugadores1:
        estadisticas_ofensivas = getEstadisticasOfensivas(j["_id"], maximos)
        estadisticas_defensivas = getEstadisticasDefensivas(j["_id"], maximos)
        suma_total_ofensiva_e1 += estadisticas_ofensivas
        suma_total_defensiva_e1 += estadisticas_defensivas
    for j in jugadores2:
        estadisticas_ofensivas = getEstadisticasOfensivas(j["_id"], maximos)
        estadisticas_defensivas = getEstadisticasDefensivas(j["_id"], maximos)
        suma_total_ofensiva_e2 += estadisticas_ofensivas
        suma_total_defensiva_e2 += estadisticas_defensivas
    
    total_e1 = suma_total_ofensiva_e1 - suma_total_defensiva_e2
    total_e2 = suma_total_ofensiva_e2 - suma_total_defensiva_e1

    resultado = total_e1 - total_e2

    if resultado > 2:
        resultado = "local"
    elif resultado < -2:
        resultado = "visitante"
    else:
        resultado = "empate"
\end{lstlisting}

Como se describió antes, es solo la sumatoria de las estadísticas normalizadas de todos los jugadores del equipo. El equipo que saque más puntuación gana. Como se puede observar, el valor que determina dónde cortar para elegir es 2, al ser un programa simple donde no se ha utilizado ninguna técnica de marchine learning se ha puesto a mano.

Las funciones getEstadisticasOfensivas y getEstadisticasDefensivas recorren los datos del jugador y se quedan con las estadísticas correspondientes normalizadas entre 0 y 1, siendo 1 la máxima ejecución de esa estadística en el partido y 0 la menor.

\begin{figure}[H]
    \centering
    \includegraphics[width=1.0\textwidth]{plantilla-TFG-ETSIIT/doc/imagenes/Ejecucion_simple.png}
    \caption{Ejecución algoritmo sin zonas}
    \label{fig:etiqueta-imagen}
\end{figure}

Ejecutando todos los partidos de la base de datos sale una precisión del 36.14\%, algo que se puede considerar bajo ya que solo hay 3 resultados posibles de un partido: gana local, gana visitante o empate; por lo que esta predicción es casi lo mismo que hacerlo al azar ya que cada uno tiene una probabilidad de un 33\% de salir.

\section{Algoritmo con pesos}
Se va a proceder a implementar el algoritmo explicado en la sección anterior. Como se indicó, se ha decidido dividir el terreno de juego en 24 zonas y cada una tendrá un peso. La división del campo en zonas es la siguiente:

\begin{figure}[H]
    \centering
    \includegraphics[width=0.8\textwidth]{plantilla-TFG-ETSIIT/doc/imagenes/Zonas_campo.png}
    \caption{Distribución en zonas del campo}
    \label{fig:etiqueta-imagen}
\end{figure}

Al igual que antes, el campo se ve de izquierda a derecha, siendo las zonas 6, 12, 18 y 24 las de máximo ataque y 1, 7, 13, 19 las más defensivas.

Para este problema inicial se ha implementado 2 vectores de pesos, uno para las estadísticas constructivas (de ataque) y otro para las destructivas (de defensa), aunque en el código se ha tratado como un único vector realmente.

La separación de vectores se hace para evaluar correctamente a cada estadística, pues no tendría sentido ponderar las acciones defensivas con un peso bajo al no ser una zona de ataque. El vector de ataque va aumentando gradualmente conforme las zonas se acercan al área rival y el vector de defensa al contrario, va aumentando conforme te acercas a tu propia portería.

De esta manera obtenemos los 2 vectores siguientes. El vector inicial para las estadísticas ofensivas:

\begin{figure}[H]
    \centering
    \includegraphics[width=0.8\textwidth]{plantilla-TFG-ETSIIT/doc/imagenes/Pesos_ini_ataque.png}
    \caption{Vector inicial de ataque}
    \label{fig:etiqueta-imagen}
\end{figure}

Y el vector inicial para las estadísticas defensivas:

\begin{figure}[H]
    \centering
    \includegraphics[width=0.8\textwidth]{plantilla-TFG-ETSIIT/doc/imagenes/Pesos_ini_defensa.png}
    \caption{Vector inicial de defensa}
    \label{fig:etiqueta-imagen}
\end{figure}

Una vez se han definido los pesos de cada zona se va a pasar a definir la implementación del algoritmo. Primero se tienen que inicializar los datos de los jugadores de ese partido.

\begin{lstlisting}[language=Python, caption={Inicialización datos}, label={lst:codigo-python}]
def calcula_zonas(local, visitante, temporada, corte=2):
    if not JUGADORES_CACHE:
        inicializar_cache_jugadores()

    # Obtener el partido
    partido = resultados.find_one({"Local": local, "Visitante": visitante, "Temporada": temporada})
    if not partido:
        raise ValueError(f"No se encontro el partido entre {local} y {visitante} en la temporada {temporada}.")

    jugadores1 = [jug for jug in JUGADORES_CACHE.values() if jug["equipo"] == local and jug["rival"] == visitante and jug["temporada"] == temporada]
    jugadores2 = [jug for jug in JUGADORES_CACHE.values() if jug["equipo"] == visitante and jug["rival"] == local and jug["temporada"] == temporada]
    
    preprocesar_mapa_calor(jugadores1)
    preprocesar_mapa_calor(jugadores2)

    maximos = calcularMaximos(jugadores1, jugadores2)

\end{lstlisting}

Primero se inicializan los jugadores en la caché, esto se hace para disminuir las consultas a la base de datos. De esta forma se buscan todos los jugadores una vez y se guardan en una variable llamada $\text{JUGADORES\_CACHE}$.

Después, se busca el partido con una consulta y se inicializan las variables con los jugadores deseados. Se procesan los mapas de calor, donde se calcula el porcentaje que ha estado el jugador en cada zona. Por último, se calculan los máximos que hay de cada estadística y se guarda en una variable local.

\begin{lstlisting}[language=Python, caption={Procesamiento de cada zona}, label={lst:codigo-python}]
for z in zonas[:-1]:
        suma_total_ofensiva_e1 = 0
        suma_total_defensiva_e1 = 0
        suma_total_ofensiva_e2 = 0
        suma_total_defensiva_e2 = 0
        for j in jugadores1:
            porc = calcularPorcentaje(j, z)
            if porc != 0:
                estadisticas_ofensivas = getEstadisticasOfensivas(j["_id"], maximos) * porc
                estadisticas_defensivas = getEstadisticasDefensivas(j["_id"], maximos) * porc
                suma_total_ofensiva_e1 += estadisticas_ofensivas
                suma_total_defensiva_e1 += estadisticas_defensivas

        for j2 in jugadores2:
            porc = calcularPorcentaje(j2, z)
            if porc != 0:
                estadisticas_ofensivas = getEstadisticasOfensivas(j2["_id"], maximos) * porc
                estadisticas_defensivas = getEstadisticasDefensivas(j2["_id"], maximos) * porc
                suma_total_ofensiva_e2 += estadisticas_ofensivas
                suma_total_defensiva_e2 += estadisticas_defensivas

\end{lstlisting}

El código hace lo siguiente, por cada zona del campo se recorren los jugadores de ambos equipos, por cada jugador de cada equipo se extraen sus estadísticas y se multiplican por el porcentaje de influencia del jugador en esa zona según el mapa de calor. Las estadísticas ya están normalizadas y se añaden directamente al total de su equipo.

\begin{lstlisting}[language=Python, caption={Procesamiento de cada zona}, label={lst:codigo-python}]
    zonas_valores_e1[z-1] = (suma_total_ofensiva_e1*zonas_coeficientes[z-1] - suma_total_defensiva_e2*zonas_coeficientes[z-1+24])
    
    zonas_valores_e2[z-1] = (suma_total_ofensiva_e2*zonas_coeficientes[z-1] - suma_total_defensiva_e1*zonas_coeficientes[z-1+24])
    
    total_rival[z-1] = (suma_total_ofensiva_e2*zonas_coeficientes[z-1] + suma_total_defensiva_e1*zonas_coeficientes[z-1+24])

suma_total = sum(zonas_valores_e1) - sum(zonas_valores_e2)

if suma_total > corte:
    return 1
elif suma_total < -corte:
    return 2
else:
    return 0

\end{lstlisting}

Una vez obtenidos el total ofensivo y defensivo de cada equipo solo queda multiplicarlo por el peso que tenga cada zona y compararlos, según el número resultante gana uno, otro o empatan.

\begin{figure}[H]
    \centering
    \includegraphics[width=1.0\textwidth]{plantilla-TFG-ETSIIT/doc/imagenes/Ejecucion_con_zonas.png}
    \caption{Ejecución algoritmo con zonas}
    \label{fig:etiqueta-imagen}
\end{figure}

Si ejecutamos todos los partidos de la base de datos con este algoritmo, obtenemos una precisión del 67.47\%, algo que ya es mucho más aceptable que antes, aunque todavía se puede mejorar más.

Una funcionalidad extra que tiene el algoritmo es que, gracias a que calcula la sumatoria total en cada zona de los 2 equipos, podemos ver qué equipos han dominado en cada zona. Esto se hace gracias a la función 'representar\_mapa', a la que se le pasan 2 argumentos que son los 2 vectores con la sumatoria total de cada equipo en cada zona del campo, de esta manera crea un mapa de 24 zonas y colorea en rojo las zonas en las que domina el equipo 1 y en azul las que domina el equipo 2. También se tienen en cuenta la simetría de los equipos, para que se siga viendo igual que hasta ahora de izquierda a derecha se ajustan las estadísticas del visitante para que también ataque de izquierda a derecha. Por lo que si el visitante domina en ataque lo veremos en la zona en la que nosotros vemos siempre el ataque, solo que de color azul porque es el visitante.

Un ejemplo de este mapa en el Villarreal - Las Palmas de la liga española temporada 2024/2025:

\begin{figure}[H]
    \centering
    \includegraphics[width=0.7\textwidth]{plantilla-TFG-ETSIIT/doc/imagenes/Local_zonas_equipos.png}
    \caption{Mapa de zonas resultante del Villareal - Las Palmas}
    \label{fig:etiqueta-imagen}
\end{figure}

Como se puede apreciar el equipo local, el Villarreal, dominó totalmente las zonas de ataque, especialmente las más cerca de la portería rival, esto tiene sentido pues el Villarreal ganó este partido 3-1. De esta manera un entrenador podría sacar conclusiones sobre el partido y ver qué zonas debería mejorar, por ejemplo el equipo visitante ha recibido más ataques sobre la banda derecha, llegando incluso a línea de fondo, por lo que tendría que defender más esa zona. El entrenador del Villarreal podría llegar a la conclusión de que fue un partido muy bueno de su equipo ya que practicamente no perdieron en ninguna zona, o si lo hacen es por muy poco pudiendo considerarse casi un empate.

Vamos a ver ahora un ejemplo donde gana el equipo visitante:

\begin{figure}[H]
    \centering
    \includegraphics[width=0.7\textwidth]{plantilla-TFG-ETSIIT/doc/imagenes/Visitante_zonas_equipo.png}
    \caption{Mapa de zonas resultante del Getafe - Las Palmas}
    \label{fig:etiqueta-imagen}
\end{figure}

Como se puede apreciar, el equipo visitante, que representa de color azul, ha dominado en prácticamente en todas las zonas del campo, menos en las 2 zonas principales de defensa. Lo que puede llevar a la conclusión de que el equipo local se dedicó principalmente a defender, concentrando sus acciones defensivas en esas zonas, pero a pesar de sus esfuerzos no logró parar el vendaval ofensivo del rival.

Ahora vamos a ver uno en el que los 2 equipos empaten:
\begin{figure}[H]
    \centering
    \includegraphics[width=0.7\textwidth]{plantilla-TFG-ETSIIT/doc/imagenes/Empate_zonas_equipos.png}
    \caption{Mapa de zonas resultante del Las Palmas - Sevilla}
    \label{fig:etiqueta-imagen}
\end{figure}

En este ejemplo entre Las Palmas - Sevilla de la liga española de la temporada 2024/2025, ningún equipo consiguió dominar el área rival, sin embargo, el centro del campo sí fue dominado por el local, aunque esto no le sirvió para la victoria.

Esta información puede serle muy útil a los entrenadores ya que pueden ver en qué zonas el equipo debe mejorar y sacar conclusiones del resultado del partido.

\section{Rendimiento individual de cada futbolista}
En el fútbol, no solo es importante analizar al equipo, sino que también lo es hacerlo para cada jugador individualmente. De la misma manera que se pueden acceder a las características totales almacenadas por el equipo en cada zona, también se puede hacer para cada jugador individualmente. Esto nos puede servir para ver qué jugadores son los que han aportado menos en el partido o cuáles han sido los más desequilibrantes.

La manera de valorarlos es la siguiente, si la sumatoria de estadísticas de un rival en una zona es mayor o ligeramente inferior que la suma total del equipo rival en esa zona es porque el futbolista ha sido diferencial.
Vamos a ver algunos mapas de calor de unas futbolistas de la liga F al aplicar esto en el partido
Real Madrid F - Valencia F:

\begin{figure}[H]
    \centering
    \includegraphics[width=0.7\textwidth]{plantilla-TFG-ETSIIT/doc/imagenes/futbolista_mapa_1.png}
    \caption{Mapa de influencia de Athenea de Castillo}
    \label{fig:etiqueta-imagen}
\end{figure}

Este mapa indica que las zonas que hay coloreadas son los sitios donde la jugadora ha sido superior a todo el equipo rival en cuanto a estadísticas. El rojo significa que ha sido superior y el azul que ha sido ligeramente inferior, las que no están coloreadas es porque ha sido claramente inferior. Como se puede apreciar, esta futbolista que es extrema izquierda fue muy diferencial en el partido ya que dominó las zonas de su posición.

Vamos a ver este otro mapa de una lateral derecha:

\begin{figure}[H]
    \centering
    \includegraphics[width=0.7\textwidth]{plantilla-TFG-ETSIIT/doc/imagenes/Futbolista_mapa_2.png}
    \caption{Mapa de influencia de Sheila García}
    \label{fig:etiqueta-imagen}
\end{figure}

Como se puede ver en la imagen, esta futbolista fue claramente diferencial en su banda, algo que tiene sentido ya que fue valorada como una de las mejores del partido.

Vamos a ver el mapa de la extrema izquierda que entró a sustituir a la que había:

\begin{figure}[H]
    \centering
    \includegraphics[width=0.7\textwidth]{plantilla-TFG-ETSIIT/doc/imagenes/Futbolista_mapa_3.png}
    \caption{Mapa de influencia de Paula Comendador}
    \label{fig:etiqueta-imagen}
\end{figure}

Como se puede ver, esta futbolista no fue tan diferencial en el partido como la otra extremo titular que se ha visto antes. Esto le puede servir al entrenador para tomar decisiones sobre las alineaciones.

Vamos a ver un último ejemplo con una defensa central del partido Barcelona F - Athletic Club de Bilbao F:

\begin{figure}[H]
    \centering
    \includegraphics[width=0.7\textwidth]{plantilla-TFG-ETSIIT/doc/imagenes/futbolista_mapa_4.png}
    \caption{Mapa de influencia de Leire Baños}
    \label{fig:etiqueta-imagen}
\end{figure}

En este partido, se puede deducir que la central por la derecha hizo un mal partido ya que en las zonas del centro de la defensa no consigue ganar, solo un poco en los laterales, por lo que el entrenador podría deducir que la jugadora ha tenido que salir de su posición para llegar a las ayudas en defensa para las laterales.

Estas son interpretaciones que podría hacer un entrenador de fútbol, interpretando situaciones del juego como las coberturas, la presión alta o los juegos por las bandas, incluso podrían llegar a ser mucho más profundas interpretando el estilo de juego del rival. Es otra manera de ver cómo con esta técnica el entrenador puede llegar a preparar mejor un partido.

\section{Algoritmo genético}

Para implementarlo, definimos al vector de ataque y defensa como un único vector que será considerado como un miembro de la población. Tendrá un tamaño de 48, ya que hay que tener en cuenta las ponderaciones de ataque y de defensa, estas ponderaciones tendrán un valor entre 0 y 1.

\subsection*{Función de evaluación}
Para medir la calidad de un individuo se utiliza la función 'evaluar\_vector' a la que se le pasa como argumento el vector de coeficientes y lo ejecuta en todos los partidos de la base de datos, dando así el porcentaje de acierto que tiene ese vector. Cuanto mayor sea ese porcentaje de aciertos mayor será la calidad del individuo.

\subsection*{Operadores genéticos}

Para la selección se ha optado por una selección por torneo. En \cite{algoritmo_genetico}, podemos ver cómo de precisa es esta técnica, ya que mezcla aleatoriedad con la selección de los mejores individuos.

\begin{lstlisting}[language=Python, caption={Selección por torneo}, label={lst:codigo-python}]
toolbox.register("select", tools.selTournament)
def algoritmo_genetico():
    ...
    offspring = toolbox.select(poblacion, len(poblacion) - 1,TAMANO_TORNEO)

\end{lstlisting}

El cruzamiento se hace con una probabilidad muy alta para favorecer la variedad de los individuos, la tasa de cruzamiento en las distintas ejecuciones que se han realizado ha sido de 0.9 ó 1. Una vez decidido qué 2 vectores se cruzan, lo harán de manera uniforme y con una probabilidad del 50\% cada gen.

\begin{lstlisting}[language=Python, caption={Cruzamiento}, label={lst:codigo-python}]
toolbox.register("mate", tools.cxUniform, indpb=0.5)

def algoritmo_genetico():
        ...
        for i in range(1, len(offspring), 2):
            if np.random.rand() < TASA_CRUZAMIENTO:
                toolbox.mate(offspring[i-1], offspring[i])
\end{lstlisting}

La mutación se aplica con una probabilidad de 1 / (tamaño de población) a cada gen, es decir, a cada elemento del vector de pesos.
    
\begin{lstlisting}[language=Python, caption={Función mutación}, label={lst:codigo-python}]
def mutacion_uniforme_flotante(individuo, low=0.0, up=1.0, indpb = INDPB_MUTACION):
    for i in range(len(individuo)):
        if np.random.rand() < indpb:
            individuo[i] = np.random.uniform(low, up)
    return individuo,
\end{lstlisting}

\begin{lstlisting}[language=Python, caption={aplicación función mutación}, label={lst:codigo-python}]
toolbox.register("mutate", mutacion_uniforme_flotante, low=0.0, up=1.0, indpb=INDPB_MUTACION)

def algoritmo_genetico():
    ...
    for ind in offspring:
            toolbox.mutate(ind)
\end{lstlisting}

\subsection*{Elitismo}
Para asegurar que la mejor solución no se pierda, se implementa elitismo que conserva al mejor individuo de cada generación.

\begin{lstlisting}[language=Python, caption={Implementación elitismo}, label={lst:codigo-python}]
def algoritmo_genetico():
    ...
    elite = tools.selBest(poblacion, k=1)
    poblacion[:] = elite + offspring

\end{lstlisting}

\subsection{Ejecución y resultados}
Para poder explorar las distintas soluciones del problema, y para asegurarnos de que nos quedamos con la mejor, se va a ejecutar varias veces combinando los datos:

\begin{itemize}
    \item Población: El de tamaño de la población será de 40 u 80.
    \item El número de iteraciones que hace el algoritmo será de 20, 40 u 80.
    \item La tasa de mutación será de 1 ó 0.9.
\end{itemize}

De esta manera, primero se ejecutará el algoritmo de población 40, con 20 iteraciones y 1 de tasa de mutación, después igual pero con 0.9 de tasa de mutación y así hasta cubrir todas las posibilidades. Esto tarda en ejecutarse varias horas por lo que también se ha visto interesante recolectar el tiempo que ha tardado cada ejecución.

Para asegurarnos de que el resultado es fiable y no solo una mera coincidencia de la aleatoriedad, se debería de ejecutar cada combinación 30 veces, sin embargo, por falta de tiempo se hará solo 5 veces, garantizando así un mínimo de garantías. No se mostrarán todas las tablas sino la media de las 5 ejecuciones junto con la desviación típica de estas.

\begin{table}[H]
\centering
\caption{Resultados del algoritmo genético para diferentes configuraciones}
\label{tab:resultados_algoritmo}
\begin{tabular}{|c|c|c|c|c|c|}
\hline
\textbf{Población} & \textbf{Generaciones} & \textbf{Cruzamiento} & \textbf{Mutación} & \textbf{Precisión (\%)} & \textbf{ejecución (min)} \\
\hline
40 & 20 & 1 & 1/48 & $85.02 \pm 0.67$  & 10\\
40 & 20 & 0.9 & 1/48 & $84.32 \pm 1.43$ & 10 \\
40 & 40 & 0.9 & 1/48 & $83.34 \pm 2.46$ & 19 \\
40 & 40 & 1 & 1/48 & $85.02 \pm 0.65$ & 19 \\

80 & 20 & 0.9 & 1/48 & $84.78 \pm 0.65$ & 19\\
40 & 20 & 1 & 1/48 & $85.98 \pm 1.07$ & 19 \\
40 & 80 & 1 & 1/48 & $85.5 \pm 1.2$& 36 \\
40 & 80 & 0.9 & 1/48 & $84.3 \pm 3.39$ & 36 \\

80 & 40 & 0.9 & 1/48 & $86.46 \pm 0.53$ & 36\\
40 & 40 & 1 & 1/48 & $85.98 \pm 0.65$ & 36 \\
80 & 80 & 1 & 1/48 & $86.94 \pm 0.53$ & 69 \\
80 & 80 & 0.9 & 1/48 & $86.46 \pm 0.53$ & 69 \\

\hline
\end{tabular}
\end{table}

Como se puede ver en la tabla, la mejor combinación es la de 80 individuos, 80 generaciones y tasa de cruzamiento de 1 con una precisión del 86.94\%. Vamos a ver un gráfico de la evolución del algoritmo con esos parámetros.

\begin{figure}[H]
    \centering
    \includegraphics[width=1.0\textwidth]{plantilla-TFG-ETSIIT/doc/imagenes/Precision_gen.png}
    \caption{Gráfico algoritmo genético}
    \label{fig:etiqueta-imagen}
\end{figure}

Como se puede ver en el gráfico, el algoritmo evoluciona muy rápido, sin embargo, se estanca a partir de la generación 24. Los vectores que ha dado como resultado han sido los siguientes:

\begin{figure}[H]
    \centering
    \includegraphics[width=1.0\textwidth]{plantilla-TFG-ETSIIT/doc/imagenes/Ataque_genetico.png}
    \caption{Vector de ataque genético}
    \label{fig:etiqueta-imagen}
\end{figure}

\begin{figure}[H]
    \centering
    \includegraphics[width=1.0\textwidth]{plantilla-TFG-ETSIIT/doc/imagenes/Defensa_genetico.png}
    \caption{Vector de defensa genético}
    \label{fig:etiqueta-imagen}
\end{figure}

\section{Algoritmo genético con $\delta$}
Para intentar que la predicción sea lo más precisa posible, vamos ahora a añadir al vector de pesos a $\delta$, que es la variable que representa en qué momento cortar para decidir si gana uno, otro o hay empate. Hasta ahora esta variable había sido añadida a mano como 2, es decir si:

\begin{itemize}
    \item $Score global > 2$, gana local
    \item $Score global < -2$, gana visitante
    \item empate en caso contrario
\end{itemize}

Ahora el algoritmo genético, aparte de definir los vectores de ataque y defensa para las ponderaciones de las zonas del campo, también va a decidir esta variable. La configuración del algoritmo es igual a la que se ha explicado antes, excepto para la tasa de mutación, que al añadir un nuevo valor al vector ha pasado a ser de 1/49. El objetivo es comparar técnicas y ver cuál es más rentable en cuanto a precisión y recursos.

Los distintos valores con los que se va a probar son los mismos que antes, se ha ejecutado un total de 5 veces y se ha calculado la media de cada combinación y su desviación típica.

\begin{table}[H]
\centering
\caption{Resultados del algoritmo genético con $\delta $para diferentes configuraciones}
\label{tab:resultados_algoritmo}
\begin{tabular}{|c|c|c|c|c|c|}
\hline
\textbf{Población} & \textbf{Generaciones} & \textbf{Cruzamiento} & \textbf{Mutación} & \textbf{Precisión (\%)} & \textbf{ejecución (min)} \\
\hline
40 & 20 & 1 & 1/49 & $83.58 \pm 1.07$  & 13\\
40 & 20 & 0.9 & 1/49 & $83.6 \pm 1.03$ & 13 \\
40 & 40 & 0.9 & 1/49 & $83.58 \pm 0.65$ & 23 \\
40 & 40 & 1 & 1/49 & $85.02 \pm 1.37$ & 23 \\

80 & 20 & 0.9 & 1/49 & $85.5 \pm 1.47$ & 24\\
40 & 20 & 1 & 1/49 & $85.02 \pm 1.37$ & 24 \\
40 & 80 & 1 & 1/49 & $85.04 \pm 2.21$& 41 \\
40 & 80 & 0.9 & 1/49 & $84.78 \pm 1.37$ & 41 \\

80 & 40 & 0.9 & 1/49 & $86.22 \pm 0.65$ & 41\\
40 & 40 & 1 & 1/49 & $86.94 \pm 0.53$ & 41 \\
80 & 80 & 1 & 1/49 & $86.96 \pm 0.58$ & 86 \\
80 & 80 & 0.9 & 1/49 & $87.2 \pm 1.38$ & 86 \\

\hline
\end{tabular}
\end{table}

Como se puede apreciar, el mejor es el último con un 87.2 de precisión media y 1.38 de desviación típica. El gráfico al ejecutarlo es el siguiente:

\begin{figure}[H]
    \centering
    \includegraphics[width=1.0\textwidth]{plantilla-TFG-ETSIIT/doc/imagenes/Precision_delta.png}
    \caption{Grafico algoritmo genético}
    \label{fig:etiqueta-imagen}
\end{figure}

El gráfico es muy parecido al anterior, empieza evolucionando rápido, sin embargo, a partir de la generación 18 se estanca y no consigue avanzar más. Los vectores resultado son:

\begin{figure}[H]
    \centering
    \includegraphics[width=1.0\textwidth]{plantilla-TFG-ETSIIT/doc/imagenes/Ataque_gen_delta.png}
    \caption{Vector de ataque genético con $\delta$}
    \label{fig:etiqueta-imagen}
\end{figure}

\begin{figure}[H]
    \centering
    \includegraphics[width=1.0\textwidth]{plantilla-TFG-ETSIIT/doc/imagenes/Defensa_gen_delta.png}
    \caption{Vector de defensa genético con $\delta$}
    \label{fig:etiqueta-imagen}
\end{figure}

\section{Conclusión}
Realmente no hay mucha diferencia entre un algoritmo y otro, por lo que se puede llegar a la conclusión de que el punto en el que decidir cortar el score global para definir el ganador no es determinante, seguramente porque cada partido tenga unos valores más grandes o pequeños según su desarrollo, y la puntuación de los equipos se adaptan a las circunstancias, por lo que los partidos no siguen una tendencia fija de score global.

A parte de eso, el porcentaje de acierto que se puede llegar a alcanzar ($87.2 \pm 1.38$) es bastante elevado para un partido de fútbol, por lo que se puede decir de manera satisfactoria que se ha cumplido el objetivo inicial de este TFG.


	% Trabajos futuros

        \chapter{Conclusiones y trabajos futuros}

\section{Conclusiones}

En conclusión, este TFG tenía como objetivo la predicción de partidos de fútbol mediante el uso de machine learning, haciendo uso de una base de datos con las estadísticas y mapas de calor de los futbolistas, dándonos una visión espacial del rendimiento de cada futbolista. Se ha conseguido una precisión de hasta un 87.2\%, algo que se puede considerar exitoso ya que el fútbol es un deporte muy impredecible.

Para completar este objetivo general se ha tenido que cumplir con la realización de los objetivos específicos indicados en el capítulo 1:

\begin{itemize}
    \item OE1: Revisión de la literatura. Se ha completado al 100\%, buscando artículos de calidad sobre fútbol y machine learning obteniendo un estado del arte completo y que trata varios ámbitos dentro de este campo. Se puede consultar en el capítulo 2, Estado del arte.

    \item OE2: Diseño de la base de datos. Se ha completado al 100\%, se ha definido una base de datos completa en MongoDB que cubría todas las necesidades del proyecto. Se puede consultar en el capítulo 4, Análisis y diseño.

    \item OE3: Obtención de las estadísticas de los jugadores. Se ha completado al 100\%, se ha implementado un programa para la inserción de los datos y se han sido obtenidos correctamente. Se puede consultar en el capítulo 6, Implentación.

    \item OE4: Diseño del algoritmo. Se ha completado al 100\%, se ha diseñado un algoritmo para determinar el ganador de un partido dividiendo el campo en zonas y utilizando las estadísticas almacenadas de los jugadores, así como ver que equipos ha dominado en cada zona. Se puede consultar en los capítulos 4 y 5, de Análisis e Implementación respectivamente.

    \item OE5: Implementación de machine learning. Se ha completado al 100\%, se han implementado 2 algoritmos genéticos, uno para el vector de pesos y otro para determinar además en que punto cortar para determinar al ganador, obteniendo resultados casi idénticos en ambos. Se puede consultar en el capítulo 6, Implementación

    \item OE6: Obtención del rendimiento individual de cada jugador del partido. Se ha completado al 100\%, se han usado las estadísticas del partido para ver la influencia de cada futbolista de manera individual y obtener así un análisis mucho más profundo para ver que futbolistas son mas diferenciales. Se puede consultar en el capítulo 6, Implementación.
\end{itemize}

En este TFG también se ha apoyado la inclusión del fútbol femenino, formando parte de la base de datos y analizando a algunas jugadoras en partidos concretos como en el capítulo 5, Implementación.

Como valoración personal, estoy satisfecho con este TFG ya que el fútbol es un deporte que me apasiona y que sigo asiduamente. Tampoco me ha costado mucho estar involucrado con el proyecto  por esos motivos, además de que he aumentado mis conocimientos futbolísticos. En cuanto a lo académico, he aprendido sobre los diferentes mecanismos de machine learning, especialmente de los algoritmos genéticos, que nunca los había dado en la carrera. También he aprendido a organizarme mejor en el tiempo y a hacer el código de más calidad y no tan desordenado. En conclusión, estoy satisfecho con el trabajo realizado y con lo aprendido.

\section{Trabajos futuros}
Aunque se hayan cumplido todos los objetivos de este TFG, tanto específicos como el general, todavía hay muchas opciones para continuar desarrollando.

Por ejemplo, en este proyecto se muestra la actuación individual de cada jugador para que el entrenador pueda analizarlo, sin embargo, se podría hacer que una máquina lo hiciera por él, generando la mejor formación y alineación posible según el rival. También se podría hacer más preciso, dividiendo el partido en varias partes de 15 minutos por ejemplo, y analizar el rendimiento del equipo en cada una de ellas. Esto es algo que ocurre a menudo en el fútbol moderno, un entrenador puede ordenar a su centrocampista por la izquierda que haga coberturas durante la primera media hora de partido porque el rival siempre sale presionando por ese lado, y a partir de esa media hora ayudar a construir en el centro del campo. Teniendo en cuenta esto, una máquina podría recomendar la mejor posición de cada jugador en cada período de tiempo para que el entrenador lo ordene en ese partido.

En cuanto a la precisión de la predicción, se podría llegar a simular un partido con agentes, esto es, teniendo en cuenta las estadísticas de los futbolistas como pase, tiro, regate. Simular un partido que refleje unas características parecidas pero siempre con un rango de aleatoriedad. Aunque sí es cierto que esto es algo que ya hacen algunos videojuegos como Top Eleven, \cite{Top_eleven}, que simulan partidos de equipos e incluso se pueden ver a los jugadores moverse durante el partido, pero se podría intentar hacer de una manera mucho más precisa y con fundamento matemático.

Por último, ya que se tienen almacenadas las estadísticas de los jugadores, se podría ampliar la base de datos y hacer un sistema de recomendación de fichajes para los equipos. Por ejemplo, mediante un algoritmo de machine learning se analizan los puntos débiles del equipo, digamos el lateral derecho, y automáticamente busque en la base de datos el mejor lateral derecho que cumpla con las características necesarias. Así, si un equipo que por ejemplo juege por las bandas tiene un lateral que pone malos centros, lo podría detectar y recomendarle uno con una alta tasa de centros. Todo esto lo podría detectar el algoritmo detectando puntos débiles en el conjunto de características del equipo.


	
	\newpage
	\bibliography{bibliografia}
	\bibliographystyle{plain}
	
\end{document}

